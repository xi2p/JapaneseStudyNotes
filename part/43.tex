\chapter{標準日本語 第43課}

摘要:\textbf{样态}

\separatorline

\section{文法}

\subsection{「よう」表样态}

\begin{codeblock}
・・・ようです

・・・のようです
\end{codeblock}

相当于\textbf{“好像・・・”},表示推测。\textbf{这种推测一般是主观、没有太大把握的},\textbf{没有很具体的根据}。

\textbf{「よう」是名词性质的,其接续方式与名词一致}。前面应该接普通形。

\textbf{前接名词时,需要加「の」。}前接形容动词时,去掉「だ」。

\begin{example}
    \begin{itemize}
        \item 今日は 雨が \textbf{降る ようです}。
        \item 張さんは 元気が\textbf{ない ようです}。
        \item あの人は \textbf{田中さんの ようです}。
    \end{itemize}
\end{example}


\separatorline

\subsection{「らしい」表样态}

\begin{codeblock}
・・・らしいです
\end{codeblock}

相当于\textbf{“好像・・・”},表示推测。\textbf{这种推测一般是相对客观},\textbf{具有一定的依据}。

「らしい」前面应该接普通形。\textbf{前接名词时,不需要加「の」。}前接形容动词时,去掉「だ」。

\begin{example}
    \begin{itemize}
        \item 電気が 消えていますから、 王さんは 部屋に\textbf{いない らしいです}。
        \item 王さんに 聞きましたが、 この店の料理は \textbf{美味しい らしいです}。
        \item 新聞で 読みましたが、 事件が \textbf{起こった らしいです}。
    \end{itemize}
\end{example}

\separatorline

\subsection{「そう」「よう」「らしい」的区别}

本小节主要三种语法的表义区别。对于语法结构/接续方式,请参见前文。

\subsubsection{「そう」表传闻}

「そう」表示\textbf{“听说・・・”},表示传闻。这是对他人说的话的\textbf{客观}转述,而\textbf{不是自己的推测}。

\subsubsection{「そう」表样态}

「そう」表示\textbf{“好像・・・”},表示推测。

一般地,这种推测是\textbf{基于直接的视觉的感官信息}。此语法是根据视觉推测,却不能推测视觉本身。

\begin{example}
    \begin{itemize}
        \item この料理は 美味し\textbf{そうです}。
        
        这道菜看起来很好吃。(根据菜的外观等视觉信息推测)

        \item \sout{この服は 赤そう です。}
        
        \sout{这件衣服看起来是红色的。}(视觉直接能看到颜色,不必推测)

    \end{itemize}
\end{example}

此外,「そう」接在动作性动词后,可以表示对将来的\textbf{一种模糊不清的预感或预想}。

\begin{example}
    \begin{itemize}
        \item 誰か 来\textbf{そうです}。(好像有人要来了。)
        \item 明日は 行け\textbf{そうです}。(明天也许能去。)
    \end{itemize}
\end{example}

\textbf{「そう」不能用于推测已经发生(过去)的事情},只能用于推测现在或将来的事情。推测过去的事情时,应该使用「よう」或「らしい」。

\subsubsection{「よう」表样态}

表示样态的「よう」可以用来叙述依据\textbf{五官或者身体的感觉}做出的推测。

相较其他的两种语法,\textbf{「よう」推测的把握小一些,更加主观,不太具备依据。}

一般地,视觉更倾向用「そう」表示,听觉更倾向用「らしい」表示,但其他感觉(嗅觉、味觉、触觉)只能用「よう」表示。

\begin{example}
    \begin{itemize}
        \item このお風呂 \textbf{温かい ようです}。(触觉)
        
        \item 誰が\textbf{来たようです}、 声が聞こえます。(听觉)
        
        \item この料理は \textbf{辛い ようです}。(味觉)
    \end{itemize}
\end{example}

除了表示样态,「よう」还可以表示\textbf{比喻},相当于“像・・・一样”。此时常与「まるで」连用。

\begin{example}
    \begin{itemize}
        \item \textbf{まるで} 夢に\textbf{いる ようだ}。
        
        \textbf{像}在梦中\textbf{一样}。

        \item 彼女は \textbf{まるで} 天使\textbf{のようだ}。
        
        她\textbf{像}天使\textbf{一样}。
    \end{itemize}
\end{example}

\subsubsection{「らしい」表样态}

「らしい」是\textbf{基于外部信息做出的相对客观}的推测,多用于依据\textbf{听觉}做出的推测。

基于传闻信息的推测,只能用「らしい」,不能用「よう」。

\begin{example}
    \begin{itemize}
        \item 足音が聞こえますから、 誰か\textbf{来た らしいです}。(听觉)
        
        \item 医者の話によると、 彼は 病気が\textbf{治った らしいです}。
        
        这里是基于传闻(医生的话)的\textbf{推测},不能用「よう」。也不能用「そう」表传闻。因为「そう」表传闻不具备推测的意思。
    \end{itemize}
\end{example}

除了表示样态,「らしい」还可以表示\textbf{具备典型特征},相当于“具有・・・的典型特征”,用于修饰名词。

\begin{example}
    \begin{itemize}
        \item 彼は \textbf{男らしい}人です。
        
        他是个\textbf{很有男子气概的}人。

        \item 彼女は \textbf{女らしい}人です。
        
        她是个\textbf{很有女性气质的}人。
    \end{itemize}
\end{example}

此处的例子也印证了「らしい」前接名词时,不需要加「の」。

\separatorline

\subsection{「すぎる」表程度太甚}

\begin{codeblock}
・・・すぎます
\end{codeblock}

表示程度太甚,过于・・・。

\begin{tabularx}{\linewidth}{|X|X|}
    \hline
    词性 & 接续规则 \\
    \hline
    动词 & 动词连用形 + すぎます \\
    \hline
    一类形容词 & \textbf{一类形容词词干} + すぎます \\
    \hline
    二类形容词 & \textbf{二类形容词词干} + すぎます \\
    \hline
\end{tabularx}

\begin{example}
    \begin{itemize}
        \item 日本人は \textbf{働き すぎます}。
        
        日本人工作\textbf{过于}辛苦。

        \item この部屋は \textbf{狭 すぎます}。
        
        这个房间\textbf{太}小了。

        \item この音楽は \textbf{にぎやか すぎます}。
        
        这音乐\textbf{太}吵了。
    \end{itemize}
\end{example}

\separatorline

\newpage

\section{词语与用法}

\subsection{「によると」表根据}

\begin{codeblock}
・・・によると
\end{codeblock}

表示根据・・・,依据・・・。

\begin{example}
    \begin{itemize}
        \item 新聞\textbf{によると}、 明日は 雨が 降る そうです。
        
        根据报纸,听说明天会下雨。
    \end{itemize}
\end{example}

\separatorline

\subsection{「という」表所谓的・・・}

\begin{codeblock}
甲 という 乙(名词)
\end{codeblock}

用甲来明确地表示乙的\textbf{内容}。相当于\textbf{“所谓的・・・”}。

\begin{example}
    \begin{itemize}
        \item 「さようなら」\textbf{という}のは、 別れる時に 言う 言葉です。
        
        \textbf{所谓}“再见”,是在分别时说的话。
    \end{itemize}
\end{example}

\separatorline

\subsection{「ようとする」推量形修饰名词}

\begin{codeblock}
・・・ようとする + 名词
\end{codeblock}

前面学过推量形,表示“打算・・・,想要・・・”。当使用推量形修饰名词时,使用此结构。

\begin{example}
    \begin{itemize}
        \item 日本へ 行\textbf{こうとする}人が 増えています。
        
        \textbf{打算去日本的人}在增加。
    \end{itemize}
\end{example}


