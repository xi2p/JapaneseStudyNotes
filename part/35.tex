\chapter{標準日本語 第35課}

摘要:\textbf{無し}

\separatorline

\section{文法}

\subsection{「だろう」表推测}

\begin{codeblock}
・・・だろう と思います
\end{codeblock}

表示说话人的推测。

\textbf{「だろう」是「でしょう」的普通形}。其接续要求与「でしょう」一致。「・・・と思います」要求前续普通形,故需要使用「だろう」。

\begin{callout}
    「だろう」本身就可以表达推测,加上「と思います」更加礼貌一些,同时也更加强调这是说话人的\textbf{个人}推测。
\end{callout}

\begin{example}
    \begin{itemize}
        \item 来週は もっと 忙しく \textbf{なる だろうと思います}。
        \item 彼は きっと 図書館に \textbf{いる だろうと思います}。
        \item この工場は 将来、 もっと 機械化が \textbf{進む だろうと思います}。
    \end{itemize}
\end{example}

\separatorline

\subsection{「ていく / てくる」表移动的趋向}

\begin{codeblock}
[・・・动词て形] いく / くる
\end{codeblock}

在动词て形后接「いく / くる」,可以表示移动方位的趋向。

使用「・・・て\textbf{いき}ます」,可以表示移动主体从说话人的视线中\textbf{由近及远}(去)。

使用「・・・て\textbf{き}ます」,可以表示移动主体从远处\textbf{向说话人靠近}(来)。

\begin{example}
    \begin{itemize}
        \item 自動車が \textbf{入っていきました}。(汽车进\textbf{去}了)
        \item 自動車が \textbf{出てきました}。(汽车出\textbf{来}了)
    \end{itemize}
\end{example}

\separatorline

\subsection{「のに」表用途}

\begin{codeblock}
・・・のに、 ~
\end{codeblock}

这并非全新的语法。这是「・・・の」+「・・・に」。前者是用「の」进行指代,后者是表示目的。结合起来是为了达成・・・,做~。相当于\textbf{表示用途}。
\begin{callout}
    虽然语法结构是「の + に」,但在实际使用中,它已经凝固成一个表示目的的语法。

    日本人在讲话的时候,也不会去分析它的语法结构,而是直接理解为一个整体的表达方式。
\end{callout}

它比「ために」语气稍轻,\textbf{更偏书面或正式口语}。

\begin{example}
    \begin{itemize}
        \item 危険な作業を \textbf{するのに}、 ロボットを 使います。
        \item 本を \textbf{買うのに}、 このお金を 利用してください。
    \end{itemize}
\end{example}

\separatorline

\subsection{「て」表理由}

\begin{codeblock}
[・・・て形]、 ~
\end{codeblock}

使用て形表\textbf{原因}。前句所述是后句的原因或理由。

根据前句词性,变换て形时会有「て」「くて」「で」。此语法前些课程已有涉及。

\begin{example}
    \begin{itemize}
        \item 風邪を \textbf{ひいて}、 学校を 休みました。
        
        \textbf{因为}感冒了,\textbf{所以}没去上学。

        \item 頭が \textbf{痛くて}、 勉強が できません。
        \item 工場の機械化が \textbf{進んで}、 生産台数が 増えました。
    \end{itemize}
\end{example}

\begin{callout}
「て形」本身就带有轻微的因果含义。此前学过使用「て形」连接句子,其蕴含的轻微的因果关系就是出于此。

此外,第28所学「ので」表原因,本质就是用「の」先指代前句内容,再用「で」表示原因。不过「ので」语气更正式,更书面化。
\end{callout}