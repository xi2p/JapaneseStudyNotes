\chapter{標準日本語 第24課}

摘要:\textbf{动词短句修饰名词}

\separatorline

\section{文法}

\subsection{动词短句修饰名词}

\begin{codeblock}
[・・・普通形] ~
\end{codeblock}

使用动词短句(普通形)可以修饰名词,\textbf{短句中提示主语要用「が」}。

\begin{example}
    \begin{itemize}
        \item 机の 上に \textbf{ある 本}
        \item 山下さんが 勉強して \textbf{いる 大学}
        \item 中国へ 行ったこと \textbf{ない 人}
        \item 去年 中国へ \textbf{行った 人}は 山田さん です。
        \item 中国は 長い 歴史を \textbf{持つ 国}です。
        \item 中国を \textbf{旅行する 人}は 大勢 います。
    \end{itemize}
\end{example}

\begin{callout}
    「は」一般是提示\textbf{整个句子}的主题。

    \vspace{10pt}

    动词句中提示主语要用「が」,否则短句中的主语会被提升为整个句子的主语。
\end{callout}

\begin{example}
    \begin{itemize}
        \item 窓\textbf{が} 開いている 部屋は ありますか。
    \end{itemize}
\end{example}

\separatorline

\section{词语与用法}

\subsection{去~旅行}

\begin{codeblock}
~を 旅行する
\end{codeblock}

\textbf{「を」常用于提示动作的路径},类似的还有「道を 歩く」、「川を 渡る」、「山を 登る」等。

\separatorline

\subsection{「与える」表给予}

\begin{codeblock}
~を 与える
\end{codeblock}

表示\textbf{抽象的给予},或\textbf{赋予}。

\begin{example}
    \begin{itemize}
        \item 中国の文化 は 日本の文化に 大きい \textbf{影響}を 与えました。
    \end{itemize}
\end{example}

\separatorline

\subsection{形容词变名词}

一类形容词把词尾\textbf{い}变成\textbf{さ},二类形容词则直接加上\textbf{さ}

\begin{tabularx}{\linewidth}{|X|X|}
\hline
形容词 & 名词 \\
\hline
明るい & 明るさ \\
\hline
長い & 長さ \\
\hline
雄大だ & 雄大さ \\
\hline
にぎやかだ & にぎやかさ \\
\hline
\end{tabularx}

把词尾变成\textbf{み}也可以变成名词,但是这主要强调\textbf{主观感受}方面。而变为\textbf{さ}则强调\textbf{客观属性}。

举例:悲しみ 楽しみ

\separatorline

\subsection{表示感到感动}

\begin{codeblock}
~に 感動します
\end{codeblock}

表示因为什么而感到,用\textbf{「に」}。

\begin{example}
    \begin{itemize}
        \item 雄大さ\textbf{に} 感動しました。
    \end{itemize}
\end{example}
