\chapter{標準日本語 第26課}

摘要:\textbf{使用「の」指代事/物/人}

\separatorline

\section{文法}

\subsection{「の」指代事/物/人}

\begin{codeblock}
[・・・动词句普通形]のは ・・・です

[・・・动词句普通形]のが ・・・です

[・・・动词句普通形]のを ・・・です
\end{codeblock}

可以使用「の」指代事/物/人。将动词句的动词形态变为普通形,并在后续「の」,即可把该句变为与名词作用相同的句子(名词句)

\begin{example}
    \begin{itemize}
        \item 日本語を \textbf{勉強するの}は 楽しいです。
        \item 花瓶を \textbf{壊したの}は 誰ですか。
        \item 宿題が \textbf{あったの}を 思い出した。
    \end{itemize}
\end{example}

「こと」也可以达到类似的效果。\textbf{但是「こと」只能指代事。}

此外,\textbf{「こと」更偏向指代事的内容,「の」偏向指代事情本身。}

\begin{example}
    \begin{itemize}
        \item お母さんに 話す\textbf{の}を 忘れました。(忘记和妈妈\textbf{说话}了)
        \item お母さんに 話す\textbf{こと}を 忘れました。(忘记和妈妈说的\textbf{内容}了)
    \end{itemize}
\end{example}

\separatorline

\subsection{「で」表示原因}

使用词的「で形」可以表示表示原因、理由。

\begin{example}
    \begin{itemize}
        \item 空港は 大勢の人\textbf{で} 混雑しています。
    \end{itemize}
\end{example}

\separatorline

\subsection{「目」表示次序}

\begin{codeblock}
~目
\end{codeblock}

相当于汉语的“第~”,接在数量词后面,表示事物的顺序。

\begin{example}
    \begin{itemize}
        \item 二度目
        \item 五年目
    \end{itemize}
\end{example}

特别地,对于"第一次",要说成\textbf{「初めて」}。
