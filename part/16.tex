\chapter{標準日本語 第16課}
 
摘要:\textbf{动词て形的一些语法}

\separatorline

\section{文法}

%%%%%%%%%%%%%%%%%%%%%%%%%%%%%%%%%%%%%%%%%%%%%%%%%%%%%%%%%%%%%%%%%%%%%%%%%%%%%%%%

\subsection{「てもいい」表示可以做某事}

\begin{codeblock}
[动词て形] + も いいです
\end{codeblock}

直接翻译是“就算做~也可以”,即表示可以做某事。



\begin{example}

    \begin{itemize}
        \item 風呂に \textbf{入っても いいです}。(可以洗澡。)
    \end{itemize}

\end{example}

\separatorline

%%%%%%%%%%%%%%%%%%%%%%%%%%%%%%%%%%%%%%%%%%%%%%%%%%%%%%%%%%%%%%%%%%%%%%%%%%%%%%%%

\subsection{「てはいけません」表示不可以做某事}

\begin{codeblock}
[动词て形] + は いけません
\end{codeblock}



\begin{example}

    \begin{itemize}
        \item ここで 本を \textbf{読んでは いけません}。(这里不可以看书。)
    \end{itemize}

\end{example}

\separatorline


%%%%%%%%%%%%%%%%%%%%%%%%%%%%%%%%%%%%%%%%%%%%%%%%%%%%%%%%%%%%%%%%%%%%%%%%%%%%%%%%

\subsection{「て」表示动作先后发生}

\begin{codeblock}
[动词て形]、 ~
\end{codeblock}

表示动作先后发生。即使不是动作紧挨在一起也可以用此句型,只要时间上是先后关系即可。



\begin{example}

    \begin{itemize}
        \item 昨日 デパートへ \textbf{行って}、 買い物\textbf{して}、 映画を\textbf{見ました}。
    \end{itemize}

\end{example}

此语法可以级联。句子中间的动词就用て形,不用考虑时态。

\separatorline

%%%%%%%%%%%%%%%%%%%%%%%%%%%%%%%%%%%%%%%%%%%%%%%%%%%%%%%%%%%%%%%%%%%%%%%%%%%%%%%%

\subsection{「てから」表示动作相继发生}

\begin{codeblock}
[动词て形]から、 ~
\end{codeblock}

此语法相对于 语法(3),更强调相继。动作紧接着发生。



\begin{example}

    \begin{itemize}
        \item 手を\textbf{洗ってから}、食事をしてください。(请洗了手就来吃饭。)
        \item テレビを\textbf{見てから}、勉強します。
    \end{itemize}

\end{example}