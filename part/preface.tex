% preface.tex
\chapter*{前言}

\parindent=2em

关于我为什么会写这份资料的问题,这还可以追溯到一个CC98一路楼...

\begin{callout}
    lz这已经是第三次自学日语了,前两次都在中途夭折了。这一次lz发誓一定要坚持下去学习日语,早日认定N2!

    \vspace{10pt}

    说回前两次学日语,第一次是在高中。我们高中有发教学用平板,里面正好有日语教材。于是lz在高二,使用劣质的平板(画面刷新很慢,触摸常常失效),坚持学完了初级上。然而高三忙碌,遂放弃。
    
    \vspace{10pt}
    
    第二次是在大一寒假,lz又打算重拾日语,遂在b站上跟着“阿飞老师”重新从零开始学日语。这次学的比较扎实,但可惜初级上还没学完就开学了。开学后,我还自学了一课,之后就没下文了。
    
    \vspace{10pt}
    
    现在,我再一次决定开始学日语。不同与以往,这次我选择全程跟随任洁老师的日语课(智云课堂)学习。我曾经觉得智云课堂上的太慢了,但现在我认为这才是最适合我的自学方式。
    
    \vspace{10pt}
    
    在先前几次学日语的基础上,我倍速播放,过完了日语Ⅰ的大部分课程,还剩下9节课。开学在即,我希望开学后我仍能坚持学习日语Ⅱ,日语Ⅲ,遂开帖,供我自己记录日语学习心得,也供8u监督。
\end{callout}


\begin{callout}
    lz先前在游玩《樱花,萌放》时,因为翻译的不正确,导致多处理解困难(虽然漆原雪人的文字本来就有点难理解)。
    
    \vspace{10pt}
    
    lz希望有朝一日可以不再借助汉化组的力量,自给自足完成阅读理解。
    
    \vspace{10pt}
    
    lz发誓,一定会用生肉形式,游玩《五彩斑斓的世界》。我会一直把他留在我的硬盘里,直到我有能力自己读懂他。
\end{callout}

笔者在学习日语过程中,每学一节课,就会在\textbf{第二天}做这节课的笔记。
通过这种方式,来对学习的内容进行复习。不过,最初的笔记是用Markdown写的,后来为了排版美观,才转成了LaTeX。
把Markdown改写成LaTeX的过程(不是借助AI一键转换的),对于笔者来说也算是第二次复习,并且最终复习效果还是不错的。于是,这本笔记就这样诞生了。
\newpage

\chapter*{前言}

关于这本书的内容,本书主要涵盖了《标准日本语 初级 上下册》(旧版)中1-45课的大部分语法点。对应着浙江大学的日语Ⅰ、日语Ⅱ、日语Ⅲ课程内容。

其中,关于日语Ⅰ的内容,由于笔者当时觉得太简单,就没怎么做笔记。所以,日语Ⅰ的内容比较少。

相比之下,日语Ⅱ、日语Ⅲ的内容就比较多了。但仍有个别课程没有笔记。

本书在一定程度上是对标日书本的摘抄,但是也增加了笔者自己的一些理解和说明,增加了任老师上课补充的一些内容,增加了目录便于查找语法点。
可以说,本书是标日的缩略版+任老师课堂内容+笔者个人理解。

\separatorline

这份资料里使用的一些术语可能并不统一,在此说明一下:

一类动词 = 第一类动词 = 五段动词

二类动词 = 第二类动词 = 一段动词

三类动词 = 第三类动词

变形 = 活用

\newpage

\chapter*{前言}

关于日语自学,笔者有几点建议:

首先是借助什么资料学习。
笔者推荐使用任洁老师的智云课堂课程,配合《标准日本语 初级上下册》(旧版)教材。
任老师讲的真的很细致,适合自学。如果你觉得她讲得太慢,可以倍速播放。
\textbf{在这里笔者需要强调:你绝对不可能只使用本资料就能学会日语。}本资料的定位只是“辅助资料”。

其次是学习的频率。笔者建议一个星期学习不超过3课(标日书本的课)。实际上笔者就是用这个频率学习的。
任老师的一次智云课堂是90分钟,两次智云课堂一般就是1课的内容。所以,一个星期学习3课的话,就代表每星期看6次智云课堂,一天一次。

然后是复习。复习是很重要的,笔者每次都在学完一节课的第二天写笔记。写笔记的过程也是复习的过程。
不复习真的会忘。

接着是背单词相关,笔者在学习的过程中没有刻意背过单词。
原因是,上面提到过,学完1课,要看180分钟的智云课堂视频。
在这180分钟内,课堂上会反复出现这一课的单词。笔者光是看完智云,就已经大概记住了这些单词。
不过,如果不复习的话,还是过几天就会忘掉的。在这里,笔者推荐一个App叫Moji辞書,用来背单词特别方便。
上课学语法和背单词其实是可以稍微分离的。

最后,如果你是一名老二次元,那么笔者认为你学习日语是很有优势的。
这里的优势不仅体现在对日语的兴趣、对一些表达的熟悉上,更体现在“正反馈”上。
学习是需要正反馈的,他能给人以成就感,让人知道自己在进步。
每当你学到一个新语法时,你就会想到:“哦,这个语法我在某某动漫的台词里听过!”。
这种正反馈会极大地激励你继续学习下去。

\begin{callout}
    笔者在学到「そうすれば」这个表达时,立刻想到了「さくら、もゆ」中くろ的台词:

    \vspace{10pt}

    \begin{quote}
        \textbf{「そうすれば、全部元通りになるはずだから」}
    \end{quote}

    \vspace{10pt}
    
    这句话的意思是“那样的话,一切都会恢复原状的”。

    \vspace{10pt}
    
    笔者当即感到高兴不已——“我能听懂くろ的台词了!”。正是这种源源不断的正反馈,支撑着笔者坚持学习日语。相信各位老二次元也能体会到这种感觉吧!
\end{callout}

祝各位学日语顺利!
