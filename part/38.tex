\chapter{標準日本語 第38課}

摘要:\textbf{动词授受关系}

\separatorline

\section{文法}

\subsection{动词授受关系}

\begin{codeblock}
[・・・て] あげます/もらいます/くれます
\end{codeblock}

表示动词授受关系,把\textbf{前句て形动作赋予某人},要注意授受动词的内外关系。

授受动词不止这些,下列出更加完整的版本:

% 使用固定宽度居中列并支持纵向合并
\renewcommand\tabularxcolumn[1]{m{#1}}
\newcolumntype{M}[1]{>{\centering\arraybackslash}m{#1}}% 居中且垂直居中的定宽列
\begin{tabularx}{\linewidth}{|M{0.3\linewidth}|X|X|X|}
    \hline
    类别 & 表达 & 意思 & 敬意 \\
    \hline
    \multirow{3}{*}{\parbox[c]{0.8\linewidth}{\centering (给予)内は $\to$ 外に}} & ~てさしあげる & 给 & 高 \\
    \cline{2-4}
    & ~てあげる & 给 & 中 \\
    \cline{2-4}
    & ~てやる & 给(动物、植物) & 低 \\
    \hline
    \multicolumn{4}{|c|}{} \\
    \hline
    \multirow{2}{*}{\parbox[c]{0.8\linewidth}{\centering (收获)内は $\to$ 外に}} & ~ていただく & 收到 & 高 \\
    \cline{2-4}
    & ~てもらう & 收到 & 中 \\
    \hline
    \multicolumn{4}{|c|}{} \\
    \hline
    \multirow{2}{*}{\parbox[c]{0.8\linewidth}{\centering (收获)外は $\to$ 内に}} & ~てくださる & 给我方 & 高 \\
    \cline{2-4}
    & ~てくれる & 给我方 & 中 \\
    \hline
\end{tabularx}


\begin{example}
    \begin{itemize}
        \item 先生は 純子さんたちに 本を \textbf{読んで あげます}。(老师\textbf{给}纯子她们\textbf{读书})

        \item 王さんが 家へ 教えに \textbf{来て くれました}。(把来家里教这个动作给我)

        \item 田中さんは 王さんに 太極拳を \textbf{教えて もらいました}。
    \end{itemize}
\end{example}

\begin{callout}
    要注意て形句中\textbf{动词是谁发出的}。例如:

    \vspace{10pt}

    \begin{itemize}
        \item 私は 田中さんに 本を\textbf{貸して} もらいました。
    \end{itemize}

    \vspace{10pt}

    主语是我,借出书动作是\textbf{田中发出的},是田中\textbf{借出}给我,要用「貸す」。(我获得了田中“借出书”的动作)
\end{callout}

\subsection{「动词て」表伴随状态}

\begin{codeblock}
[・・・动词て]、 ~

[・・・动词ないで]、 ~
\end{codeblock}

表示后句是在前句的状态下进行的,而\textbf{不是}简单解释为先做前句动作,再做后句动作。

在前句表示\textbf{否定}的情况下,要使用\textbf{「で」}。

\begin{example}
    \begin{itemize}
        \item 新聞を \textbf{持って} 電車になりました。(带着报纸坐上了电车)
        \item 傘を \textbf{持たないで} 出かけていきます。
    \end{itemize}
\end{example}

\separatorline

\subsection{「ために」表理由、原因}

\begin{codeblock}
・・・ために、 ~

[・・・名词]のために、 ~
\end{codeblock}

表理由、原因。接续要求与「ために」表目的时一致。

\begin{example}
    \begin{itemize}
        \item 雨が\textbf{降らない ために} 作物が 育ちません。
        \item 事故が\textbf{あった ために} 会社に 遅れました。
        \item \textbf{運動不足 の ために} 体の調子が よくありません。
    \end{itemize}
\end{example}

\separatorline

\section{词语与用法}

\subsection{「そこで」と「それで」}

「そこで」表示前项发生,\textbf{于是,在这种情况下,・・・}。前后不一定有因果关系。

「それで」表示因为前句的情况,而发生后句。有因果关系。

\separatorline

\subsection{「比べて」表比较}

\begin{codeblock}
・・・と比べて、 ~
\end{codeblock}

表示\textbf{与・・・相比}。“・・・”处是比较的对象。助词用「と」或「に」都可以。

\begin{example}
    \begin{itemize}
        \item 去年\textbf{と比べて}、 今年の 夏は 暑いです。(与去年相比,今年的夏天很热)
        \item 大阪は、 東京\textbf{に比べて}、 冬が 暖かいです。
    \end{itemize}
\end{example}

