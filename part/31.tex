\chapter{標準日本語 第31課}

摘要:\textbf{無し}

\separatorline

\section{文法}

\subsection{「つもり」打算做某事}

\begin{codeblock}
・・・つもりです
\end{codeblock}

表示要做某事的意志,相当于\textbf{“打算・・・”}。

「つもり」之前的动词用\textbf{基本形}或\textbf{ない形}。

\begin{example}
    \begin{itemize}
        \item 夏休みは \textbf{旅行する つもりです}。
        \item 明日から、 日本語の勉強を \textbf{始める つもりです}。
        \item 田中さんは 元旦に 初詣に \textbf{行く つもりです}。
    \end{itemize}
\end{example}

\separatorline

\subsection{「ことにする」表(自己)决定做某事}

\begin{codeblock}
・・・ことにします
\end{codeblock}

相当于“(因自己的意志而做出的)\textbf{决定・・・}”,表示要做某事的意志。

「こと」之前的动词用\textbf{基本形}或\textbf{ない形}。

\begin{example}
    \begin{itemize}
        \item 頭が 痛いので、 今日は 会社を \textbf{休む ことにします}。
        
        因为头很痛,今天\textbf{(我)}决定请假不去公司。

        \item 今度の 日曜日は 図書館で \textbf{勉強する ことにします}。
        
        \item 風を 引いたので、 旅行に \textbf{行かない ことにします}。
    \end{itemize}
\end{example}

\separatorline

\subsection{「ことになる」(他人)决定做某事;成为某种结果}

\begin{codeblock}
・・・ことになります
\end{codeblock}

表示某\textbf{事物被决定}或\textbf{成为某种结果}。此处的“决定”\textbf{并非出于自己的意志},而是出于某个团体的决定。

「こと」之前的动词用\textbf{基本形}或\textbf{ない形}。

\begin{example}
    \begin{itemize}
        \item 電車で \textbf{行く ことになります}。(结果决定要坐电车)
        \item この工場は \textbf{閉鎖する ことになりました}。(这个工厂决定关闭了)
        \item 来月は 試験を \textbf{しない ことになりました}。(下个月决定不考试了)
    \end{itemize}
\end{example}

\begin{callout}
    此处有一种微妙的语感。\textbf{事物被决定}和\textbf{成为某种结果}这两个含义是相辅相成的,不是独立的。

    \vspace{10pt}

    事物被决定,本身就是一种结果;而某种结果的出现,往往是因为事物被决定了。

    \vspace{10pt}

    例如,「電車で行くことになります」,既是“决定要坐电车”,也是“结果是要坐电车”。

\end{callout}

\separatorline

\subsection{「そう」表传闻}

\begin{codeblock}
・・・そうです
\end{codeblock}

\textbf{相当于“据说・・・”},表示从别人处听说某事。

「そう」之前用\textbf{普通形}。(例:「学生\textbf{だ}」「きれい\textbf{だ}」)

\begin{example}
    \begin{itemize}
        \item 北京は \textbf{寒い そうです}。
        \item 王さんは 日本の初詣の様子を \textbf{見たい そうです}。
        \item あの人は 北京大学の \textbf{学生だ そうです}。
        \item 山下さんは テニスが \textbf{上手だ そうです}。
    \end{itemize}
\end{example}

\separatorline

\subsection{「でしょう」表征求对方认同}

\begin{codeblock}
・・・でしょう
\end{codeblock}

\textbf{接续方法与表达推测的「でしょう」相同。}基本上接普通形,而对于名词和形容动词的现在肯定形,不用加「だ」

发音时,语调会有上扬,表示征求对方的认同。

\begin{example}
    \begin{itemize}
        \item あそこに 山が \textbf{見える でしょう}。
        \item 王さんは \textbf{学生 でしょう}。(王先生是学生对吧?)
    \end{itemize}
\end{example}
