\chapter{標準日本語 第40課}

摘要:\textbf{被动态}

\separatorline

\section{被动态活用}

\begin{tabularx}{\linewidth}{|X|X|X|X|}
    \hline
    词类 & 变形规则 & 辞书形 & 被动态 \\
    \hline
    一类动词 & 结尾段う变あ段+れる & 書く & 書かれる \\
    \hline
    二类动词 & 结尾去る+られる & 食べる & 食べられる \\
    \hline
    来る & 来られる & \ & \ \\
    \hline
    する & される & \ & \ \\
    \hline
\end{tabularx}

\separatorline

\section{文法}

\subsection{被动态 1}

\begin{codeblock}
甲は 乙に ・・・(ら)れます
\end{codeblock}

表示被动,表示\textbf{“甲被乙……”}。动作的发动者使用\textbf{「に」}提示。

\begin{example}
    \begin{itemize}
        \item 純子さん\textbf{は} 先生\textbf{に} \textbf{褒められました}。(纯子被老师表扬了)
        \item 私\textbf{は} 純子さん\textbf{に} \textbf{みつけられました}。(我被纯子发现了)
    \end{itemize}
\end{example}

此外,此语法的常表示从某人处接受了一个\textbf{不好的动作}。还可以有以下表达:

\begin{itemize}
    \item 私は 雨\textbf{に} \textbf{降られました}。(我被雨淋了[接受了一个降雨的动作])

    \item 純子さんは 父\textbf{に} \textbf{死なれました}。(纯子的\textbf{父亲逝世了}[接受了一个死的动作])

    \item 私は 子供\textbf{に} \textbf{泣かられていて}、 困りました。(\textbf{孩子在哭},我很困扰[接受了孩子哭的动作])
\end{itemize}

注意:表达接受其他人的动作时,\textbf{常指不好的动作}。对于好的动作,推荐使用\textbf{「・・・てもらう」}。

\begin{example}
    \begin{itemize}
        \item 私は 王さんに 教えて \textbf{もらいました}。
    \end{itemize}
\end{example}

\separatorline

\subsection{被动态 2}

\begin{codeblock}
甲は 乙に 丙を ・・・(ら)れます
\end{codeblock}

与语法 1相同,本语法表示被动,并且\textbf{指出了动作的目的及内容(宾语)}。

\begin{example}
    \begin{itemize}
        \item 田中さんは お客さん\textbf{に} お礼\textbf{を} \textbf{言われました}。(田中先生被客人说了感谢的话)
        \item 王さんは 知らない人\textbf{に} 道\textbf{を} \textbf{聞かれました}。
        \item 田中さんは 王さん\textbf{に} 足\textbf{を} \textbf{踏まれました}。(田中先生的脚被王先生踩了)
    \end{itemize}
\end{example}

\separatorline

\subsection{「やすい/にくい」表动作容易/困难}

\begin{codeblock}
[・・・动词连用形]やすい です

[・・・动词连用形]にくい です
\end{codeblock}

「・・・やすい」表示动作是容易的,表示“很容易・・・”;

「・・・にくい」表示动作是困难的,表示“很难・・・”  “不易・・・”。

前项应使用\textbf{动词连用形},即「ます形」去掉「ます」。

\begin{example}
    \begin{itemize}
        \item この辞書は とても \textbf{使いやすい} です。(这本词典\textbf{用起来很方便})
        \item 本屋の場所は \textbf{わかりにくい} です。
    \end{itemize}
\end{example}

\separatorline

\section{词语与用法}

\subsection{「向け」表面向对象}

在\textbf{名词}后\textbf{接尾「向け」},表示面向的对象。\textbf{前面的名词是面向的对象}。整个词组的词性可看做\textbf{名词}。

\begin{example}
    \begin{itemize}
        \item \textbf{中国向け}に でサインした 洋服も あります。(\textbf{面向}中国。名词作副词接「に」)
        \item \textbf{子供向け}の 番組です。(\textbf{面向}孩子。名词接名词用「の」)
    \end{itemize}   
\end{example}

类似的,还有「向き」,表示适合的对象。

\begin{example}
    \begin{itemize}
        \item \textbf{子供向き}の 番組です。(\textbf{适合}孩子的节目)
    \end{itemize}
\end{example}

