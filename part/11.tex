\chapter{標準日本語 第11課}
 
摘要:\textbf{無し}

\separatorline

\section{文法}

\subsection{表示人的好恶或水平高低}
\begin{codeblock}
Aは ・・・が ~です
\end{codeblock}
表示人的好恶或水平高低,「~」处填入表示好恶或水平高低的词语,例如「上手」「下手」「好き」「嫌い」。

\begin{example}
    \begin{itemize}
        \item 張さん \textbf{は} ピンポン \textbf{が 上手です。}
        \item 私 \textbf{は} 野球 \textbf{が 好きです。}
    \end{itemize}
\end{example}

\separatorline

\subsection{「わかる」表示了解,知道}
\begin{example}
    \begin{itemize}
        \item 私 \textbf{は} 日本語 \textbf{が わかります。}
    \end{itemize}
\end{example}
除了「わかる」,还可以用「知る(しる)」表示“知道”。但是两者有所区别:

「わかる」表示“懂”,强调对事物的理解。

「知る」表示“知道”,只说明知道事物的存在,不强调理解。

\separatorline

\subsection{「人気がある」表示有人气}
\begin{example}
    \begin{itemize}
        \item この花 \textbf{は} たいへん \textbf{人気があります。}
    \end{itemize}
\end{example}

例句中「たいへん」放在「人気」前。因为「人気があります」作为常用语,是一个整体。