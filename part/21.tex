\chapter{標準日本語 第21課}

摘要:\textbf{动词た形}

\separatorline

\section{文法}

\subsection{「たことがある」表做过某事}

\begin{codeblock}
・・・[た] ことが あります
\end{codeblock}

表示做过某事,\textbf{有做过某事的经验 / 记忆}。



\begin{example}
    \begin{itemize}
        \item 田中さんは 歌舞伎を \textbf{見た ことが ありません}。
        \item 私は 王さんに \textbf{会った ことが あります}。
    \end{itemize}
\end{example}

\textbf{注意!}这个语法里,无论做的某事发生在何时,都用た形,结尾用ある / あります\textbf{(与时态无关)}。

\separatorline

\subsection{「あとで」表做・・・之后,~}

\begin{codeblock}
・・・[た] あとで、 ~
\end{codeblock}



\begin{example}
    \begin{itemize}
        \item 仕事が \textbf{終わった あとで} 映画を見ます。
        \item 授業が \textbf{終わった あとで} 歌舞伎の話をします。
    \end{itemize}
\end{example}

\separatorline

\newpage 

\subsection{「たほうがいい」表做・・・比较好}

\begin{codeblock}
・・・[た] ほうが いいです 
\end{codeblock}

表示做・・・比较好,用于劝对方最好采取某种行动。



\begin{example}
    \begin{itemize}
        \item 一度 \textbf{見た ほうが いいです}。
        \item 薬を \textbf{飲んだ ほうが いいです}。
    \end{itemize}
\end{example}

\separatorline

\subsection{主动承担某事}

\begin{codeblock}
わたしが ・・・ましょう
\end{codeblock}

表示主动承担某事。说话者提议自己来做某事。



\begin{example}
    \begin{itemize}
        \item \textbf{わたしが} 切符を 買いに 行き\textbf{ましょう}。(我去买票吧)
    \end{itemize}
\end{example}
