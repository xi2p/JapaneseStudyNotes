\chapter{標準日本語 第19課}

摘要:\textbf{未然形}

\separatorline

\section{未然形活用}

\begin{tabularx}{\linewidth}{|X|X|X|X|}
\hline
动词类型 & 变形规律 & 原型示例 & 变形示例 \\
\hline
一类动词 & 结尾变あ行 + ない & 読む & 読まない \\
\hline
一类动词(以う结尾) & 结尾变わ + ない & 会う & 会わない \\
\hline
ある & ない & \ & \ \\
\hline
二类动词 & 结尾去る + ない & 食べる & 食べない \\
\hline
する & しない & \ & \ \\
\hline
くる & 来(こ)ない & \ & \ \\
\hline
\end{tabularx}

\separatorline

\section{文法}

\subsection{「ないでください」表请不要做某事}

\begin{codeblock}
・・・ [动词ない形] で ください
\end{codeblock}

表示请求对方不要做某事。

「ない」后固定用\textbf{で}ください,不用考虑て。



\begin{example}
    \begin{itemize}
        \item 煙草を \textbf{吸わないで ください}。
        \item 会社に \textbf{遅れないで ください}。
    \end{itemize}
\end{example}

\newpage

\separatorline

\subsection{「なければなりません」表不得不做某事}

\begin{codeblock}
・・・ [なければ] なりません
\end{codeblock}

表示不得不做某事,相当于英语的“have to do something”。

「なければ」是把动词的未然形结尾的「ない」变成「なければ」。



\begin{example}
    \begin{itemize}
        \item 許可を \textbf{もらわなければ なりません}。
        \item 学校に \textbf{行かなければ なりません}。
    \end{itemize}

\end{example}

\begin{callout}
    「なければなりません」看起来是一大坨,但其实他是由两部分组成的:

    「なければ」是「ない」的假定形,表示“如果不・・・”;

    「なりません」是「なる」的否定形式,表示“不成为”或“不行”。

    合起来,「なければなりません」的意思就是“如果不・・・就不行”,也就是\textbf{“必须做某事”}。
\end{callout}

在口语中,「なければなりません」常被缩略为「なきゃ」或「なくちゃ」。

\separatorline

\subsection{「なくてもいいです」表不做某事也可以}

\begin{codeblock}
・・・ [なくても] いいです
\end{codeblock}

表示不做某事也可以。「なくて」是「ない」的て形。



\begin{example}
    \begin{itemize}
        \item 会社に \textbf{行かなくても いいです}。
        \item 彼女に \textbf{言わなくても いいです}。
    \end{itemize}

\end{example}


\begin{callout}
    「なくてもいいです」这个语法的本质是之后会学的「て形」表即使:

    「なくて」是「ない」的て形,表示“即使不・・・”;

    「いい」表示“好”。

    合起来,「なくてもいいです」的意思就是“即使不・・・也可以”,也就是\textbf{“不做某事也可以”}。
\end{callout}