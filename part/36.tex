\chapter{標準日本語 第36課}

摘要:\textbf{假设,如果}

\separatorline

\section{文法}

\subsection{「たら」表如果}

\begin{codeblock}
[・・・たら]、 ~
\end{codeblock}

\begin{callout}
・・・たら 的本质是た形。根据前续成分的不同,有不同的接续要求(即た形变形规则)

\vspace{10pt}

・・・たら(动词)

\vspace{10pt}

・・・かったら(一类形容词)

\vspace{10pt}

・・・だったら(形容动词/名词)

\end{callout}

表示前句所述情况发生,就会出现后句所述情况。此时可以与「と」「ば」替换。

还可以表示\textbf{如果}前句所述情况发生,\textbf{就}会引起后句所述情况。

\textbf{「たら」后句可以使用许可、请求、推断等(意志形)}。而「と」「ば」不可。

\begin{example}
    \begin{itemize}
        \item 春に \textbf{なったら}、 花が 咲きます。
        
        (春天到了,花就会开。可以与「と」「ば」替换)

        \item 仕事が \textbf{終わったら}、 映画に 行きましょう。
        
        (后接请求,不可与「と」「ば」替换)

        \item 山田さんに \textbf{会ったら}、 この話を 伝えてください。
        
        \item 天気が \textbf{よかったら}、 出かけます。
        
        \item \textbf{雨だったら}、 野球の試合は 諦めます。
    \end{itemize}
\end{example}

\separatorline

\newpage

\subsection{「なら(ば)」表如果}

\begin{codeblock}
[・・・なら(ば)]、 ~
\end{codeblock}

可以表示\textbf{“如果,就”},后句可以使用\textbf{建议}、许可、请求、推断等(意志形)。与「たら」类似。

多用于\textbf{以对方已说的事为话题,或两个事物进行比较}。此时,虽然汉语是“如果”,但其意思与“如果发送・・・,就~”不同。

\begin{callout}
    接续要求: \textbf{普通形} + なら

    \vspace{10pt}

    对于 「形动 / 名称」,现在肯定形不用加 「だ」

    \vspace{10pt}

    结尾的「ば」是可选的,加不加意思都一样。
\end{callout}

\begin{example}
    \begin{itemize}
        \item A: テレビは どこ ですか。

        B: テレビ\textbf{なら}、 ここに あります。(\textbf{如果是}电视的话,在这里)

        \item 新聞では 朝にならなければ、 ニュースを伝えることができません。でも、 テレビ\textbf{なら}、 試合の結果が その日にわかります。
        
        (\textbf{比较}报纸和电视)

        \item 雨\textbf{ならば}、 野球の試合は 中止でしょう。

        =雨だったら、 野球の試合は 中止でしょう。(\textbf{如果}下雨,棒球的比试就终止了吧)
    \end{itemize}
\end{example}

\separatorline

\subsection{四个“如果”的比较}

% 使 X 列使用 m 型(垂直居中)。把前两列设为固定较窄宽度,第三列为可伸缩列,
% 这样第三列会比前两列更宽。
\renewcommand\tabularxcolumn[1]{m{#1}}
\newcolumntype{M}[1]{>{\centering\arraybackslash}m{#1}}% 居中且垂直居中的定宽列
\begin{tabularx}{\linewidth}{|M{0.15\linewidth}|M{0.25\linewidth}|X|}
    \hline
    语法 & 意思 & 要求 \\
    \hline
    ・・・と & 一・・・,就~ & 描述客观规律,不能接意志形,\textbf{不接过去式}。 \\
    \hline
    ・・・ば & 如果・・・,就~ & 描述客观规律,\textbf{前项是后项成立的条件}。几乎不接意志形,\textbf{不接过去式}。 \\
    \hline
    ・・・たら & 如果・・・了的话,就~ & 意义最广泛,后可以接意志形。\textbf{强调前句先发生,后句才发生。描述一次性事件}。 \\
    \hline
    ・・・なら & 如果(是)・・・,就~ & 多以对方提到的事物为话题或强调比较,可接意志形,\textbf{适合提建议}。\textbf{前后项时间顺序灵活}。 \\
    \hline
\end{tabularx}

\separatorline

\newpage

\subsection{「でも」表即使}

\begin{codeblock}
・・・でも、 ~
\end{codeblock}

「でも」前面用名词,相当于汉语的“连・・・也~”,\textbf{“即使・・・,也~”}。

\begin{example}
    \begin{itemize}
        \item \textbf{先生でも} わかりません。(\textbf{连}老师\textbf{也}不懂)
        \item 仕事が 忙しいので、 \textbf{日曜日でも} 休むことが できません。
        \item この問題は、 \textbf{子供でも} わかります。(这个问题\textbf{连}小孩子\textbf{也}会)
    \end{itemize}
\end{example}

\begin{callout}
    学到此,想必读者恍然大悟:

    \vspace{10pt}
    
    原来曾经学过的「~てもいいです」的本质是“即使・・・,也可以”,即“可以・・・”
\end{callout}

\separatorline

\section{词语与用法}

\subsection{「通して」表媒介}

\begin{codeblock}
~を通して
\end{codeblock}

表示以某人或某物传递信息或物质,“通过~”。

\begin{example}
    \begin{itemize}
        \item 友人\textbf{を通して}、 彼女に 手紙を 送りました。(\textbf{通过}朋友给她寄了信)
        \item 写真\textbf{を通して}、 知りました。(\textbf{通过}照片知道了)
    \end{itemize}
\end{example}

\separatorline

\subsection{「とか」表例举}

\begin{codeblock}
~とか、 ~とか
\end{codeblock}

类似于「や」。「とか」也可以用于不完全例举,\textbf{但只用于口语}。

「や」只用于例举名词或名词句,而\textbf{「とか」可以自由连接所有句子}。

\begin{example}
    \begin{itemize}
        \item 日曜日は、 洗濯する\textbf{とか}、 掃除する\textbf{とか}、 忙しいです。
    \end{itemize}
\end{example}

