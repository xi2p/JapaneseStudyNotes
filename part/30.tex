\chapter{標準日本語 第30課}

摘要:\textbf{無し}

\separatorline

\section{文法}

\subsection{「と」表一・・・,就~}

\begin{codeblock}
・・・と、 ~
\end{codeblock}

表示前句所述事物或现象一出现,就会引起后句所述的事物或现象。相当于“一・・・,就~(如果・・・,就~)”。

「と」之前的动词用\textbf{基本形}或\textbf{ない形}。

\begin{callout}
    这个语法常用于描述\textbf{自然规律、客观事实}等,表示某种必然的因果关系。

    \vspace{10pt}

    注意,这个语法不用于表示主观意志或计划。具体的比较,请参考后续章节。
\end{callout}

\begin{example}
    \begin{itemize}
        \item 夜に \textbf{なると}、 気温が 下がります。(自然规律)
        \item 4月に \textbf{なると}、 桜の花が 咲きます。(自然规律)
        \item 屋根や道路の 雪を \textbf{取り除かないと}、 生活することが できません。(客观事实)
    \end{itemize}
\end{example}

\separatorline

\newpage

\subsection{「ても」表即使・・・,也~}

\begin{codeblock}
[・・・动词て形]も、~
\end{codeblock}

\textbf{相当于“即使,也”}。表示如果出现了前句所述事物或现象,后句所述某事或现象也未出现,或出现了与一般情况相反的事物或现象。

\textbf{前句动词用て形。}

\begin{example}
    \begin{itemize}
        \item 辞書を \textbf{見ても}、 わかりませんでした。

        \item 薬を \textbf{飲んでも}、 風邪は よくなりません。

        \item 沖縄は、 冬に \textbf{なっても}、 雪が 降りません。
    \end{itemize}
\end{example}

\begin{callout}
    注意到曾经学过「・・・ても いいです」的句法,表示“可以・・・”。其本质是这里的「ても」,表示“即使・・・也可以”。
\end{callout}

\separatorline

\subsection{「ことがある」表有时出现某事}

\begin{codeblock}
・・・ことがあります
\end{codeblock}

表示\textbf{有时}出现某事。「こと」之前的动词用\textbf{基本形}或\textbf{ない形}。

\begin{example}
    \begin{itemize}
        \item 山田さんは 会社を \textbf{休むことが あります}。
        \item 田中さんは 朝御飯を \textbf{食べないことが あります}。(田中有时不吃早饭)
        \item 雪が降ると、 電車が 止まったり、 道路が 閉鎖に なったり \textbf{することが あります}。
    \end{itemize}
\end{example}

\begin{callout}
    \textbf{注意}此语法表示的是\textbf{有时}出现某事。如果表示\textbf{做过}某事,应使用「・・・\textbf{た} ことがあります」
\end{callout}

\separatorline

\subsection{(时间上)从・・・到・・・,断断续续有~}

\begin{codeblock}
・・・から ・・・にかけ\textbf{て}、 ~
\end{codeblock}

表示在从A时间到B时间的一段范围内,\textbf{断断续续}或持续发生某事(两者都可以表示,通过语境判断具体含义)。

\begin{example}
    \begin{itemize}
        \item 昨日 8時から 9時\textbf{にかけて}、 雨が降りました。
        
        (昨天八点到九点\textbf{断断续续}在下雨)

        \item 昨日 8時から 9時\textbf{まで}、 雨が降りました。
        
        (昨天八点到九点\textbf{一直}在下雨)
    \end{itemize}
\end{example}
