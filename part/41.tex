\chapter{標準日本語 第41課}

摘要:\textbf{無し}

\separatorline

\section{文法}

\subsection{「ように」表目的}

\begin{codeblock}
・・・ように、 ~
\end{codeblock}

前句表示\textbf{目的},后句表示手段,相当于\textbf{“为了・・・而~”}。

「ように」前接动词基本形或ない形。

\begin{example}
    \begin{itemize}
        \item 中国語が \textbf{話せる ように} 勉強しています。(为了能说中文而学习)
        \item 熱が \textbf{下がる ように} 薬を飲みました。
        \item 運動不足に \textbf{ならない ように} 注意してください。
    \end{itemize}
\end{example}

\begin{callout}
    \textbf{【辨】}「ように」和「ために」都能表示\textbf{为了},但两者有区别。

    \vspace{10pt}

    「ために」强调一种人主观意志控制的动作,前只能接\textbf{辞书形},且多为\textbf{他动词}。

    \vspace{10pt}

    「ように」强调\textbf{希望达成的状态或结果},前可以接\textbf{辞书形和ない形},且多为\textbf{自动词}。
\end{callout}

\begin{example}
    \begin{itemize}
        \item 熱が 下\textbf{が}る \textbf{ように} 薬を飲みました。
        \item 熱を 下\textbf{げ}る \textbf{ために} 薬を飲みました。
    \end{itemize}
\end{example}

\separatorline

\newpage

\subsection{「まま」表保持原状}

\begin{codeblock}
・・・[た]まま、 ~
\end{codeblock}

表示某种状态继续存在,没有变化。\textbf{保持原状}。常包含行为者本来没有这种打算,\textbf{不是出于本意}的心情。

「まま」可以看做\textbf{名词},表示“原状”,其前续原则与名词一致。\textbf{当前面接动词时,应使用「た形」}。

\begin{example}
    \begin{itemize}
        \item 靴を \textbf{はいた まま} 家に 上がりました。
        \item ラジオを \textbf{つけた まま} 寝てしまいました。(开着收音机就睡着了)
        \item \textbf{立った まま} お茶を飲みました。
    \end{itemize}
\end{example}

\separatorline

\subsection{「のに」表逆接}

\begin{codeblock}
・・・のに、 ~
\end{codeblock}

表示产生了与一般情况相反的结果,或产生了没有估计到的情况。类似于\textbf{“・・・,却~”}  "明明・・・,~"

\textbf{其接续原则与「ので」一致}。名词/形容动词前应补\textbf{「な」}。

\begin{example}
    \begin{itemize}
        \item \textbf{勉強している のに} なかなか うまくなりません。
        \item 頭が \textbf{痛かった のに} 会社に 行きました。
        \item あの人は \textbf{学生 なのに} あまり 勉強しません。
    \end{itemize}
\end{example}

\separatorline

\section{词语与用法}

\subsection{「気になる」表变得想要做某事}

\begin{codeblock}
・・・気になる
\end{codeblock}

表示“想・・・\textbf{了}”,包含一种变化,情感萌生,“开始・・・”的语感。“\textbf{开始}在意、关心、对…产生兴趣”。

前接\textbf{动词基本形}或\textbf{指示代词}「その」「そういう」「そんな」等。

\begin{example}
    \begin{itemize}
        \item デザインはいいんですが、高いので、\textbf{買う気になりません}。
        
        (设计得很好,但是太贵了,我\textbf{不想}买)

        \item \textbf{その気になって}頑張れば、すぐ覚えられますよ。
        
        (その指代的内容在上下文中)

    \end{itemize}
\end{example}

