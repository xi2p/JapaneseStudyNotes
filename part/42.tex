\chapter{標準日本語 第42課}

摘要:\textbf{使役态}

\separatorline

\section{使役态活用}

\begin{tabularx}{\linewidth}{|X|X|X|X|}

\hline
动词类型 & 变形规律 & 原型示例 & 变形示例 \\
\hline
一类动词 & 结尾变あ段 + せる & 書く & 書かせる \\
\hline
二类动词 & 去掉る + させる & 食べる & 食べさせる \\
\hline
する &  させる & 勉強する & 勉強させる \\
\hline
来る &  来させる & \ & \ \\
\hline

\end{tabularx}

\separatorline

\section{文法}

\subsection{使役态1}

\begin{codeblock}
甲は 乙を ・・・(さ)せます
\end{codeblock}

表示甲\textbf{让/使/叫}乙做某事。甲是动作的发动者,乙是动作的执行者。这种句型里多用\textbf{自动词}。

\begin{example}
    \begin{itemize}
        \item 先生は 子供たち\textbf{を} \textbf{走らせました}。(老师\textbf{让}孩子们跑步)
        \item 純子さんは 犬\textbf{を} \textbf{散歩させます}。(纯子\textbf{让}狗散步)
        \item 田中さんは 純子さん\textbf{を} 学校に \textbf{行かせました}。
    \end{itemize}
\end{example}

除了使用「を」来提示乙以外,有时也可以使用「に」:

\begin{codeblock}
甲は 乙に ・・・(さ)せます
\end{codeblock}

\begin{callout}
    使用「を」与「に」的区别在于:

    \vspace{10pt}

    「を」强调甲\textbf{强制}乙去做某事。

    \vspace{10pt}

    「に」的意思是甲允许/让乙去做某事,表达\textbf{许可/同意}的意思。
\end{callout}

\begin{example}
    \begin{itemize}
        \item 母は 子供\textbf{に} 外で 遊ばせます。(母亲允许孩子们在外面玩)
        \item 母は 子供\textbf{に} 外で 遊ばせません。(母亲不允许孩子们在外面玩)
        
        \item 先生は 子供たち\textbf{を} 走らせました。(老师让孩子们跑步)
        \item 先生は 子供たち\textbf{に} 走らせました。(老师允许孩子们跑步)
        
    \end{itemize}
\end{example}

\separatorline

\subsection{使役态2}

\begin{codeblock}
甲は 乙に 丙を ・・・(さ)せます
\end{codeblock}

表示甲\textbf{让/使/叫}乙做某事。甲是动作的发动者,乙是动作的执行者。

这种句型里一般使用\textbf{他动词}。由于存在宾语,需要使用「を」来提示丙,则\textbf{乙必须使用「に」来提示}。

\begin{example}
    \begin{itemize}
        \item お母さんは 純子さん\textbf{に} 部屋\textbf{を} \textbf{掃除させます}。
        \item 先生は 子供たち\textbf{に} 宿題\textbf{を} \textbf{させました}。
        \item 先生は わたし\textbf{に} 漢字\textbf{を} \textbf{書かせました}。
    \end{itemize}
\end{example}

\separatorline

\subsection{「間に」表动作进行的时间点}

\begin{codeblock}
・・・(の)間に、 ~
\end{codeblock}

表示在某个大时间段内的某个小时间点(段),发生了某个动作。在前句大时间段的部分时间内,发生了后句的动作。

「間」可以视为名词,其接续方式与名词一致。

\begin{example}
    \begin{itemize}
        \item わたしが 昼ご飯を 食べている\textbf{間に}、 友達が 来ました。
        
        我吃午饭的一大段时间内,朋友来了这个动作发生在这段时间的某个时间点上。

        \item わたしが テレビを 見ている\textbf{間に}、 母が 部屋を 掃除しました。
        
        我看电视的一大段时间内,妈妈打扫了房间这个动作发生在这段时间的某个时间段内。

        \item 夏休み\textbf{の間に} 東京へ 行きたいです。

    \end{itemize}
\end{example}

\separatorline

\newpage

\subsection{「間」表动作进行的时间段}

\begin{codeblock}
・・・(の)間、 ~
\end{codeblock}

表示在某个时间段内,发生了某个动作。前句的时间段内,后句的动作一直在进行。

前句动作持续多久,后句就进行多久。

\begin{example}
    \begin{itemize}
        \item わたしが 昼ご飯を 食べている\textbf{間}、 友達は 待っていました。
        
        我吃午饭的时候,朋友一直在等我。

        \item わたしが テレビを 見ている\textbf{間}、 母は 部屋を 掃除しました。
        
        我看电视的时候,妈妈打扫了房间。(一直在打扫)

        \item 夏休み\textbf{の間}、 毎日 泳ぎます。

    \end{itemize}
\end{example}

\begin{callout}
    「間に」与「間」产生区别的根源是「に」。

    \vspace{10pt}

    「に」表示在某个时间点上发生了某个动作,因此「間に」强调的是时间点,动作在时间点上进行;而「間」没有「に」,动作在时间段内进行。
\end{callout}





