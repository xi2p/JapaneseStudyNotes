\chapter{標準日本語 第22課}

摘要:\textbf{・・・[た]り、 ・・・[た]り}

\separatorline

\section{文法}

\subsection{动作列举}

\begin{codeblock}
・・・[た]り、 ・・・[た]り します/です
\end{codeblock}

用于表述\textbf{动作举例}或\textbf{动作反复进行}。有做…的,有做…的。



\begin{example}
    \begin{itemize}
        \item 田中さんは 新聞を \textbf{読んだり}、 テレビを \textbf{見たり します}。
        \item 新聞を 家で \textbf{読んだり}、 電車で \textbf{読んだり します}。
    \end{itemize}
\end{example}

\textbf{注意!}这个语法里,无论做的某事发生在何时,都用た形,结尾用します / です\textbf{(与时态无关)}。

\separatorline

\subsection{一类形容词列举}

\begin{codeblock}
[一类形]かったり、 [一类形]かったり します/です
\end{codeblock}

用于表示形容词的举例。\textbf{表示不同情况都有。既有…情况,也有…情况。} 

「一类形かった」是一类形的「た」形。



\begin{example}
    \begin{itemize}
        \item 部屋の中は \textbf{暖かかったり、 寒かったり です}。
        \item 見出しは \textbf{大きかったり、 小さかったり します}。(标题有大有小。)
    \end{itemize}

\end{example}

\separatorline

\subsection{二类形容词/名词 列举}

\begin{codeblock}
・・・だったり、 ・・・だったり します/です
\end{codeblock}

用于表示二类形容词的举例时,表示不同情况都有。既有…情况,也有…情况。

用于表示名词举例时,表示从许多事物中抽出几样举例。

「・・・だった」是 二类形容词/名词 的「た」形。



\begin{example}
    \begin{itemize}
        \item 時間に よって \textbf{静かだったり}、 \textbf{にぎやかだったり です}。(根据时间不同,有时候安静,有时候热闹)
        \item スポーツの \textbf{専門だったり}、 経済の \textbf{専門だったり します}。(有的是体育专业,有的是经济专业。)
    \end{itemize}
\end{example}

\separatorline

\begin{callout}
    不难发现,本课所有的列举句型,都是在动词/形容词/名词的「た」形后面加上「り」构成的。
    \vspace{10pt}

    「・・・かった」是一类形的「た」形。「・・・だった」是 二类形容词/名词 的「た」形。
    \vspace{10pt}

    所以这个语法的核心就是「・・・た」形+「り」,没有看上去那么复杂的变换,只要牢记各种词性的「た」形就行了。
\end{callout}

\separatorline

\section{词语与用法}

\subsection{「知っています」と「知りません」}

「知っています」用于表达知道。

「知りません」用于表达不知道。\sout{不会说「知っていません」。}

\separatorline

\subsection{~だけ}

用于表示“只”,与「しか」类似。\textbf{但「だけ」用于肯定句。}

\begin{example}
    \begin{itemize}
        \item お茶を 一杯\textbf{だけ} 飲み\textbf{ます}。
        \item お茶を 一杯\textbf{しか} 飲み\textbf{ません}。
    \end{itemize}
\end{example}