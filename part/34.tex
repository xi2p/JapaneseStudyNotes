\chapter{標準日本語 第34課}

摘要:\textbf{ば形,条件句}

\separatorline

\section{假定形(ば形)活用}

肯定形式的ば形变化规则如下:

\begin{tabularx}{\linewidth}{|X|X|X|X|}
    \hline
    原词性 & 变形规则 & 基本形 & ば形 \\
    \hline
    第一类动词 & 结尾う段变え段 + ば & 読む & 読めば \\
    \hline
    第二类动词 & 结尾る删去 + れば & 食べる & 食べれば \\
    \hline
    来る & \ & 来る & 来(く)れば \\
    \hline
    する & \ & する & すれば \\
    \hline
    一类形容词 & 词干 + ければ & 美味しい & 美味しければ \\
    \hline
    二类形容词 & 词干 + ならば(であれば) & 簡単 & 簡単ならば(簡単であれば) \\
    \hline
    名词 & 词干 + ならば(であれば) & 休み & 休みならば(休みであれば) \\
    \hline
\end{tabularx}

肯定形式的ば形表示“如果・・・”。

\separatorline

否定形式的ば形变化规则如下:

\textbf{先变否定形,再按一类形容词变}

\begin{tabularx}{\linewidth}{|X|X|X|X|}
    \hline
    原词性 & 基本形 & 否定形 & ば形 \\
    \hline
    第一类动词 & 読む & 読まない & 読まなければ \\
    \hline
    第二类动词 & 食べる & 食べない & 食べなければ \\
    \hline
    来る & 来る & 来ない & 来なければ \\
    \hline
    する & する & しない & しなければ \\
    \hline
    一类形容词 & 美味しい & 美味しくない & 美味しくなければ \\
    \hline
    二类形容词 & 簡単 & 簡単ではない & 簡単でなければ \\
    \hline
    名词 & 休み & 休みではない & 休みでなければ \\
    \hline
\end{tabularx}

否定形式的ば形表示“如果不・・・”。

\separatorline

\section{文法}

\subsection{「ば形」表如果・・・,就~}

\begin{codeblock}
[・・・ば]、 ~
\end{codeblock}

可以表示前句所述情况发生,会引起后句情况。此时,可以和「・・・と、~」互换使用。

此外,还可以表示\textbf{“如果・・・,就~”}。多用于描述\textbf{客观}条件、普遍真理、能力实现的条件。

后项多为状态、可能形、自然结果。\textbf{不可后接意志形},即不可用于劝诱、请求、命令等。

\begin{example}
    \begin{itemize}
        \item 春に \textbf{なれば}、 花が 咲きます。
        \item 今 \textbf{出発すれば}、 きっと 遅れません。
        \item 天気が \textbf{よければ}、 公園へ 行きます。
    \end{itemize}
\end{example}

\separatorline

\subsection{「なければ」表如果不・・・,就~}

\begin{codeblock}
[・・・なければ]、 ~
\end{codeblock}

是文法1的否定形式,表示“如果不・・・,就~”。接续要求与文法1一致,毕竟其本质是ば形的否定形式。

\begin{example}
    \begin{itemize}
        \item 雨が \textbf{降らなければ}、 作物は 育ちません。
        
        如果不下雨,作物就不能生长。

        \item 西瓜は、 夏に \textbf{ならなければ}、 食べることが できません。
        \item \textbf{軽くなければ}、 持つことが できません。
    \end{itemize}
\end{example}

\separatorline

\subsection{「一类形容词+くても」表即使}

\begin{codeblock}
[・・・一类形容词] くても、 ~
\end{codeblock}

与第30课所学「・・・ても、~」是同一个语法。一类形容词变「て形」的方式是「词干+くても」。

\begin{callout}
    二类形容词也是如此。其变形方式是「词干+で」。

    例: この部屋が\textbf{静かでも}、集中できません。
\end{callout}

\begin{example}
    \begin{itemize}
        \item \textbf{寒くても}、 ストーブを つけません。
        \item 天気が \textbf{悪くても}、 山に 登ろうと 思います。
    \end{itemize}
\end{example}

\separatorline

\section{词语与用法}

\subsection{「寒さに強い」}

「に」表示「強い」的对象。此处表示“抗寒性很强”。

类似的,有:暑さに強い・雨に強い・雪に強い

\separatorline

\subsection{「~による」表手段、方法、行为者}

\begin{codeblock}
・・・による、 ~
\end{codeblock}

其表示的意思随上下文改变,可表手段、方法、行为者等。

\begin{example}
    \begin{itemize}
        \item ビニルハウス による 促成栽培
        
        温室栽培

        \item 留学生 による 研究
        
        由留学生进行的研究
    \end{itemize}
\end{example}

\begin{callout}
    「よる」变「て形」,则是「よって」,表示根据。

    \vspace{10pt}

    「よる」本身作为动词,表示“顺道去……”
\end{callout}