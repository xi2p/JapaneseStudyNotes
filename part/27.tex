\chapter{標準日本語 第27課}

摘要:\textbf{・・・的时候}

\separatorline

\section{文法}

\subsection{「の時」表在做某事(名词)时}

\begin{codeblock}
[・・・名词]の時、~
\end{codeblock}

表示在做某事时,・・・

\begin{example}
    \begin{itemize}
        \item \textbf{食事の時}、 日本人は はしを 使います。
        \item 音楽の \textbf{授業の時}、 先生は ピアノを 弾きました。
    \end{itemize}
\end{example}

如果要表示在做某事 之前/之后, 则可以使用「・・・の前に」、「・・・のあとで」

\begin{example}
    \begin{itemize}
        \item この薬は、 \textbf{食事の前に} 飲んで ください。
        \item \textbf{仕事の後で}、 ビールを 飲みに 行きました。
    \end{itemize}
\end{example}

\separatorline

\newpage

\subsection{「時」表在做某事之前}

\begin{codeblock}
[・・・动词普通形]時、~
\end{codeblock}

表示在做某件事的“時”\textbf{之前}。注意动词与“時”中间不加“の”,因为动词普通形可以直接修饰名词。

\begin{callout}
    笔者对这个“之前”的表述的准确性存疑。个人认为这个句法可以表达动作发生时、以及稍稍之前一点的时间。
    
    \vspace{10pt}

    如果动作使用た形,甚至可以表达动作发生后。
\end{callout}

\begin{example}
    \begin{itemize}
        \item 日本へ\textbf{行く時}、カメラを買いました。
        
        在去日本之前 或 在去日本的途中 买了相机

        \item 食事が \textbf{終わった時}、「ごちそうさまでした。」と 言います。
        
        在吃完饭之后,说“我吃饱了”。
        
        \item 田中さんは、山に \textbf{登った時}、写真を 撮りました。
        
        田中先生在登上山的时候,拍了照片。(此时已经登上了,包含一层“登山后”的语感。)

    \end{itemize}
\end{example}

\separatorline

\subsection{「ながら」表动作同时进行}

\begin{codeblock}
[・・・动词连用形]ながら、~
\end{codeblock}

表示动作同时进行(一边・・・,一边~)。

\begin{example}
    \begin{itemize}
        \item 弟は、 ラジオを \textbf{聞きながら}、 勉強します。
        \item 子供たちは、 テレビを \textbf{見ながら}、 食事を しています。
    \end{itemize}
\end{example}

\separatorline

\newpage

\subsection{「でしょう」表推量}

\begin{codeblock}
[・・・普通形]でしょう
\end{codeblock}

表示说话人进行推测的心情。当说话人推测对方心情或预测未来事物不能明确判断时使用。

「でしょう」的读音是降调的。

对于\textbf{形容动词和名词},其现在肯定形的普通形结尾是「だ」,在接「でしょう」时,\textbf{「だ」要省略}。

\begin{example}
    \begin{itemize}
        \item 今日は 寒く\textbf{なる でしょう}。
        \item 有名な店ですから、 混んで\textbf{いる でしょう}。
        \item あの大学に 入るのは、 \textbf{難しい でしょう}。
        \item この仕事を 一人で するのは、 \textbf{たいへん でしょう}。
        \item 日曜日ですから、 あの本屋は \textbf{休み でしょう}。
    \end{itemize}
\end{example}

\separatorline

\section{词语与用法}

\subsection{「家に上がります」表进入家里}

固定用法。日本房屋地板一般比地面高一些,门口有玄关,要上台阶一样走进屋里,故用\textbf{「家に上がります」}。

\separatorline

\subsection{「にしたがって」表按照}

\begin{codeblock}
~にしたがって
\end{codeblock}

表示“按照~”的意思。“~”的部分是要遵循的规范,后句是遵循这个规范的行为。

\begin{example}
    \begin{itemize}
        \item 日本の習慣\textbf{に従って}、 「いただきます。」と 言いました。
    \end{itemize}
\end{example}