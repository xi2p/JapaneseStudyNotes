\chapter{標準日本語 第45課}

摘要:\textbf{無し}

\separatorline

\section{文法}

\subsection{「し」表列举}

\begin{codeblock}
・・・し、 ・・・し、 ~
\end{codeblock}

可以使用此句型\textbf{列举}几个事物或现象。这和「・・・たり、 ・・・たり」、「~もいれば(あれば), ~もいます(あります)」类似。

\textbf{「し」前接普通形。}

\begin{example}
    \begin{itemize}
        \item 田中さんも\textbf{行ったし}、 王さんも行きました。
        
        田中先生去了,王先生也去了。
        
        \item この町は交通が\textbf{便利だし}、 物価も\textbf{安いし}、 住みやすいです。
        
        这个城市交通便利,物价也便宜,适合居住。

        \item 家庭の仕事には、 育児も\textbf{あるし}、 炊事も\textbf{あるし}、 洗濯もあります。
    \end{itemize}
\end{example}

除了简单列举,此句型还可以进行\textbf{各种含义的衔接},如并列、递进、因果等。

\begin{example}
    \begin{itemize}
        \item 彼は頭が\textbf{いいし}、 スポーツも\textbf{できるし}、 何でもできます。
        
        他头脑聪明,而且运动也行,什么都会。\textbf{(递进)}

        \item 今日は天気が\textbf{いいし}、 時間も\textbf{あるし}、 散歩に行きましょう。
        
        今天天气好,而且时间也有,我们去散步吧。\textbf{(因果)}

        \item 電気も\textbf{消えているし}、 鍵も\textbf{かかっているし}、 王さんは部屋にはいないでしょう。
        
        电灯也关了,门也锁上了,王先生应该不在房间里吧。\textbf{(推测)}
    \end{itemize}
\end{example}

\separatorline

\newpage

\subsection{「ていく /てくる」表事态的发展}

\begin{codeblock}
[・・・て]いく

[・・・て]くる
\end{codeblock}

\begin{callout}
    此前已经介绍过「ていく / てくる」表示空间方向的移动。这里介绍表示\textbf{事态的发展变化}。
\end{callout}

「ていく / てくる」可以表示\textbf{经过时间的推移},事态的\textbf{发展}。

「ていく」表示\textbf{从现在起,向将来发展}的变化。(从现在去未来)

「てくる」表示\textbf{从过去到现在,已经发生的变化}。(从过去来现在)

\begin{example}
    \begin{itemize}
        \item 社会に進出する女性は だんだん\textbf{増えてきました}。
        
        进入社会的女性逐渐增多了。\textbf{(过去到现在的变化)}

        \item これからも ますます\textbf{増えていく}でしょう。
        
        今后也会越来越多吧。\textbf{(从现在到将来的变化)}
    \end{itemize}
\end{example}

\separatorline



\section{词语与用法}

\subsection{「続く」「続ける」作动词结尾词}

在动词连用形后接「続く」「続ける」,可以构造新的词,表示原动作的\textbf{持续}。

其中,「続く」前\textbf{只能接无意志动词},表示动作\textbf{自然持续},不受主语意志控制。而「続ける」无限制。

\begin{example}
    \begin{itemize}
        \item 雨が\textbf{降り続く} (雨\textbf{持续}下)
        \item 勉強\textbf{し続ける} (\textbf{持续}学习)
    \end{itemize}
\end{example}

\separatorline

\subsection{「直す」作动词结尾词}

在动词连用形后接「直す」,可以构造新的词,表示\textbf{重新做某事}。

\begin{example}
    \begin{itemize}
        \item 字が汚いので、 \textbf{書き直し}てください。
        
        字迹很乱,请\textbf{重新}写一遍。

        \item なにか別の方法はないか \textbf{考え直して}みよう。
        
        我们\textbf{再}想想有没有什么别的方法吧。
    \end{itemize}
\end{example}



