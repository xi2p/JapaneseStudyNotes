\chapter{標準日本語 第44課}

摘要:\textbf{無し}

\begin{callout}
    本课涉及到的一些语法,已经在前面的课里出现过。这里省略了重复的内容。
\end{callout}

\separatorline

\section{文法}

\subsection{「・・・ば ・・・ほど」表程度成正比}

\begin{codeblock}
[・・・ば形] ・・・ほど、 ~
\end{codeblock}

表示\textbf{“越・・・越~”},表示后句的变化程度和前句的程度成正比。

第一处「・・・」用假定形,第二处「・・・」用普通形。两处是相同的词语(不限于动词)。

\begin{callout}
    这个语法不能直译成“如果・・・的话,越・・・,就~”。

    \vspace{10pt}

    这体现了日语表达方式和汉语表达方式的差异。此语法应记住。
\end{callout}

\begin{example}
    \begin{itemize}
        \item \textbf{読めば} 読む\textbf{ほど} 俳句の 面白さが わかります。
        
        越读越能体会到俳句的有趣之处。

        \item \textbf{勉強すれば} する\textbf{ほど} 日本語は 上手に なる。
        
        \item 年を\textbf{取れば} 取る\textbf{ほど} 勉強したいことが 増えます。
        
        \item ワイシャツは \textbf{安ければ} 安い\textbf{ほど} いいです。(形容词)
        
        白衬衫越便宜越好。
    \end{itemize}
\end{example}

\separatorline

\newpage

\subsection{「もいれば / もあれば」表既有・・・又有・・・}

\begin{codeblock}
・・・もいれば、・・・もいます

・・・もあれば、・・・もあります
\end{codeblock}

表示\textbf{“既有・・・又有・・・”}。「ある」「いる」的用法与前面学过的相同。

此句型与「・・・もいます、 ・・・もいます」这种句型表达的意思基本相同。

\begin{example}
    \begin{itemize}
        \item 日本には 大きい町\textbf{もあれば}、 小さい町\textbf{もあります}。
        
        \item このクラスには 日本人\textbf{もいれば}、 中国人\textbf{もいます}。
        
        \item 寒い日\textbf{もあれば}、 温かい日\textbf{もあります}。
    \end{itemize}
\end{example}

\separatorline

\section{词语与用法}

\subsection{「とともに」表一起}

\begin{codeblock}
・・・とともに
\end{codeblock}

副词,表示\textbf{“和・・・一起”}。用于把两个以上事物作为一个整体来看待。

\begin{example}
    \begin{itemize}
        \item 仕事\textbf{とともに}、 余暇も大切です。
        
        \item 物の豊かさ\textbf{とともに}、 心の豊かさを 求めるのも 大切なことです。
        
        和物质的丰富一起,追求心灵的丰富也是重要的事情。
    \end{itemize}
\end{example}

\separatorline

\subsection{「に関して」关于}

\begin{codeblock}
・・・に関して
\end{codeblock}

表示\textbf{“关于・・・”}。类似于「について」。

\begin{example}
    \begin{itemize}
        \item 余暇\textbf{に関して} 日本人の意識は ずいぶん変わりました。
        
        关于余暇,日本人的意识已经发生了很大变化。
    \end{itemize}
\end{example}
