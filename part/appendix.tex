\chapter*{附录1・用言与体言}

% 使用固定宽度居中列并支持纵向合并
\begin{tabularx}{\linewidth}{|c|X|}
    \hline
    类别 & 成分 \\
    \hline
    \multirow{2}{*}{用言} & 动词 \\
    \cline{2-2}
    & 形容词 \\ % 此处同步了下文的修改
    \hline
    \multirow{3}{*}{体言} & 名词 \\
    \cline{2-2}
    & 代名词 \\
    \cline{2-2}
    & 数词 \\
    \hline
\end{tabularx}

\textbf{用言},即“是有活用的词”。日语中,动词和形容词都属于用言,因为它们都有各种活用形态,可以根据语法需要进行变化。
例如,可以变成:未然形、连用形、终止形、连体形、假定形等。

\textbf{体言}:体言是没有活用(即词尾变化)的独立词。
体言可以后续助词が(は、も等)构成主语,这是体言最大的特点。

\textbf{连用形}:即连接用言的形态。一个词变成连用形后,就可以连接用言,从而实现修饰用言等功能。
例如,动词「書く」的连用形是「書き」,可以用于连接其他动词或形容词,如「書き続ける」「書きやすい」。

\textbf{连体形}:即连接体言的形态。用言变成连体形后,可以修饰体言,从而实现对名词等的修饰功能。
例如,动词「読む」的连体形之一「読んだ」,可以用于修饰名词,如「読んだ本」(读过的书)。

\newpage

\chapter*{附录2・日语的各种形}

\textbf{辞书形/原形}:单词最基本的形态,通常用于查字典。例如,「食べる」(吃)、「行く」(去)。

\separatorline

\textbf{基本形}:日语有五大态,即主动态(原型)、被动态、使役态、可能态、自然发生态。
将辞书形变换到这五大态的形态,称为基本形。
例如:行く(原型)、行ける(可能态)、行かれる(被动态)、行かせる(使役 态)、行かせられる(使役被动态)

\textbf{注1}:基本型不包括用言活用型的过去形式。例如:行った、行かなかった、不属于基本型。

\textbf{注2}:对于“基本形”这个概念,不同的语法书有不同的定义和范围。此处所写的是符合旧标日、任老师课堂、本资料语法的定义。

\separatorline

\textbf{敬体/礼貌体/丁宁体}:即带有礼貌「ます」「です」的形式。

\begin{tabularx}{\linewidth}{|c|c|c|c|c|}
\hline
词性 & 现在肯定 & 现在否定 & 过去肯定 & 过去否定 \\
\hline
一类动词 & 行きます & 行き\textbf{ません} & 行き\textbf{ました} & 行き\textbf{ませんでした} \\
\hline
二类动词 & 食べます & 食べ\textbf{ません} & 食べ\textbf{ました} & 食べ\textbf{ませんでした} \\
\hline
形容词 & 高いです & 高\textbf{くありません} & 高\textbf{かったです} & 高\textbf{くありませんでした} \\
\hline
形容动词 & 静かです & 静か\textbf{じゃないです} & 静か\textbf{でした} & 静か\textbf{じゃなかったです} \\
\hline
名词 & 学生です & 学生\textbf{ではないです} & 学生\textbf{でした} & 学生\textbf{ではなかったです} \\
\hline
\end{tabularx}

其中,「じゃないです」=「ではないです」,是「では」的口语形式。下同。

\separatorline

\textbf{普通形/简体}:即不带有礼貌「ます」「です」的形式。(下表中除了「です」以外均为普通形/简体)

\begin{tabularx}{\linewidth}{|c|c|c|c|c|}
\hline
词性 & 现在肯定 & 现在否定 & 过去肯定 & 过去否定 \\
\hline
一类动词 & 行く & 行か\textbf{ない} & 行\textbf{った} & 行か\textbf{なかった} \\
\hline
二类动词 & 食べる & 食べ\textbf{ない} & 食べ\textbf{た} & 食べ\textbf{なかった} \\
\hline
形容词 & 高い & 高く\textbf{ない} & 高\textbf{かった} & 高く\textbf{なかった} \\
\hline
形容动词 & 静か\textbf{だ} & 静か\textbf{じゃない} & 静か\textbf{だった} & 静か\textbf{じゃなかった} \\
\hline
名词 & 学生\textbf{だ} & 学生\textbf{ではない} & 学生\textbf{だった} & 学生\textbf{ではなかった} \\
\hline
\multicolumn{5}{|c|}{} \\
\hline
です & \textbf{だ} & \textbf{ではない} & \textbf{だった} & \textbf{ではなかった} \\
\hline
\end{tabularx}

\separatorline

不难发现,一个词可能属于多种形态。例如,动词「行く」可以是辞书形、基本形、普通形等。

\textbf{注:}动词ます形,て形,ば形,意志形,不属于动词辞书型,基本型,普通型的范围。
