\chapter{標準日本語 第23課}

摘要:\textbf{普通形的使用}

\separatorline

\section{文法}

\subsection{「思います」表认为}

\begin{codeblock}
わたしは [・・・普通形]と 思います
\end{codeblock}

用于表达“我想…/我认为…”。注意,\textbf{这表达的是"I think …"而不是“欲望”}。

助词「と」用于提示内容。



\begin{example}
    \begin{itemize}
        \item わたしは 新しい 技術を \textbf{勉強したいと 思います}。
        \item 科学技術は もっと \textbf{進歩すると 思います}。
    \end{itemize}
\end{example}

其中第一句例句,直译为“我认为我想要学习新的技术”。这是一种委婉的说法。一般不会只说「わたしは新しい技術を勉強したいです」,这听起来像小孩子任性的表达。

\textbf{注意!}此语法只用于\textbf{第一人称陈述和第二人称询问}:

\begin{example}
    \begin{itemize}
        \item あなたは 今日は いい 天気\textbf{だと 思いますか}。(第二人称询问)
    \end{itemize}
\end{example}

若要表示第三人称的认为,可以使用如下语法:

\begin{codeblock}
[・・・普通形]と 思っています

[・・・普通形]と 思いました
\end{codeblock}

\begin{example}
    \begin{itemize}
        \item 張さんは たくさんの ことを 勉強したいと \textbf{思っています}。
    \end{itemize}
\end{example}

\separatorline

\subsection{「言います」表说}

\begin{codeblock}
甲は (乙に) [・・・普通形]と 言います
\end{codeblock}

表示“甲对乙说・・・”。这个语法有多种用法:

1. \textbf{间接引用}别人的话时,要保证使用普通形。

\begin{example}
    \begin{itemize}
        \item 張さんは 中国に \textbf{帰りたいと 言いました}。
    \end{itemize}
\end{example}

2. \textbf{直接引用}别人的话时,使用「 ・・・ 」将引用内容括起来,\textbf{引用内容保持说话原文不变}。

注意!被引用的句子的句号要放在引号内,不能丢弃!!!!!!

\begin{example}
    \begin{itemize}
        \item 王さんは 「今日は暑い\textbf{です}ね。」\textbf{と 言いました}。
    \end{itemize}
\end{example}

3. 表示把・・・叫做・・・

\begin{codeblock}
・・・を ・・・と 言います
\end{codeblock}

\begin{example}
    \begin{itemize}
        \item 日本人は これを 机\textbf{と 言います}。(日本人把这个叫做桌子)
        \item わたしは 田中\textbf{と 言います}。(我叫田中)
        \item \textbf{【补充】} わたしは 田中と \textbf{申します}。(我叫田中 更正式的说法)
    \end{itemize}
\end{example}

\separatorline

\subsection{「ために」表为了}

\begin{codeblock}
[・・・基本形]ために、 ~
\end{codeblock}

用于表达“为了・・・(的目的)而做~”

\begin{example}
    \begin{itemize}
        \item 友達に \textbf{会う ために}、 東京へ 行きました。
        \item 科学技術を \textbf{勉強する ために}、日本に 留学しました。
    \end{itemize}
\end{example}

另外,名词后面也可以接「~の ために」

\newpage

\separatorline

\subsection{「代わりに」表作为・・・的代替}

\begin{codeblock}
・・・の 代わりに、 ~
\end{codeblock}

用于表达“作为・・・的代替而做~”

\begin{example}
    \begin{itemize}
        \item 今日は 雨だったので、 車の \textbf{代わりに}、 電車で 行きました。
        \item 危険な仕事や 単純な仕事を \textbf{人間の代わりに} ロボットが しているのです。
    \end{itemize}
\end{example}

\separatorline

\subsection{「の(ん)です」表陈述语气}

\begin{codeblock}
・・・の(ん)です
\end{codeblock}

用于\textbf{表达陈述语气},或者\textbf{强调自己的主张},也可以\textbf{用来催促对方回答}。

使用「の」是相对正式的说法,「ん」是口语中更常用的形式。

前面接续普通形。

\textbf{注:连接 名词/形容动词 的现在肯定形时,要在前面加上な}

\begin{example}
    \begin{itemize}
        \item ロボットの 利用は 日本が 世界で いちばん \textbf{多いんです}よ。
        \item ここは \textbf{綺麗なのです}。
        \item これは \textbf{2階建てなのだ}。
        \item 今朝 どうして \textbf{遅れたのです}か。(催促对方回答今天早上为什么迟到)
    \end{itemize}
\end{example}
