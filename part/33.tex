\chapter{標準日本語 第33課}

摘要:\textbf{自他动词}

\separatorline

\section{文法}

\subsection{「ておく」表事先做好准备}

\begin{codeblock}
[・・・て] おきます
\end{codeblock}

表示事先做好某种准备。\textbf{强调此动作是为某个目的做准备}。

使用「おきます」可以表示现在或将来做准备,动作还没做。

使用「おきました」可以表示准备已经做好了,动作已发生。

\begin{example}
    \begin{itemize}
        \item 明日の朝 六時に 出かけます。 今夜 準備をして おきましょう。
        
        明天早上六点出发。今晚先做好\textbf{准备}吧。

        \item 明日は試験ですから、 勉強しておいてください。
        
        因为明天有考试,请好好学习(为了\textbf{准备}考试)。

        \item お正月用の餃子は 大みそかに 作っておきました。
        
        春节用的饺子在除夕夜就做好\textbf{了}(为了除夕\textbf{准备}的)。
    \end{itemize}
\end{example}

\separatorline

\subsection{「てしまう」表动作、作用全部结束}

\begin{codeblock}
[・・・て] しまいます
\end{codeblock}

表动作、作用全部结束。有时附有无可挽回、感到\textbf{遗憾}的心情。

\begin{example}
    \begin{itemize}
        \item この本は もう \textbf{読んで しまいました}。
        \item その本の内容は \textbf{忘れて しまいました}。
        \item バスの中に 傘を \textbf{忘れて しまいました}。(把伞忘在公交车上了,感到遗憾)
    \end{itemize}
\end{example}

\separatorline

\subsection{「そう」表样态}

\begin{codeblock}
・・・ そうです
\end{codeblock}

相当于\textbf{“好像・・・”}。表示根据\textbf{周围的状况}或\textbf{事物的外观}进行推断,感到\textbf{好像是这样},或表达\textbf{认为有这种可能性}。

根据前续的成分,有不同的接续要求。

\begin{tabularx}{\linewidth}{|X|X|X|X|}
    \hline
    前续成分 & 接续要求 & 前续示例 & 接续示例 \\
    \hline
    动词 & 连用形+そうです & 書く & 書きそうです \\
    \hline
    一类形容词 & 去掉い+そうです & 高い & 高そうです \\
    \hline
    二类形容词 & 词干+そうです & 静か & 静かそうです \\
    \hline
    いい & よさそうです & \ & \ \\
    \hline
    ない & なさそうです	& \ & \ \\
    \hline
\end{tabularx}

\begin{example}
    \begin{itemize}
        \item 雨が \textbf{降りそうです}。 傘を 持っていった ほうが いいですよ。
        \item 今日は 忙しから、 帰るのが 遅く\textbf{なりそうです}。
        \item 外は \textbf{寒そうです}。コートを 着たほうが いいですよ。
        \item みんな とても \textbf{元気そうです}。
    \end{itemize}
\end{example}

「・・・そう」的整体还可以用于修饰,可以被视为\textbf{二类形容词}。修饰名词+な。修饰动词+に。

\begin{example}
    \begin{itemize}
        \item \textbf{美味しそうな} ケーキを 食べます。
        
        (吃看起来\textbf{美味的蛋糕})

        \item \textbf{美味しそうに} ケーキを 食べます。
        
        (看起来\textbf{很美味地吃}蛋糕)
    \end{itemize}
\end{example}

\separatorline
