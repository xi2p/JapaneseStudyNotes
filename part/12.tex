\chapter{標準日本語 第12課}
 
摘要:\textbf{比较级}

\separatorline

\section{黏着语 与 比较级}
\begin{callout}

\textbf{日语是黏着语,句子通过添加助词来丰富其含义。}如果删去助词短语,句子仍能表达出大致意思(但是意思可能异于原句)。由此可以方便记忆比较级句式。

\vspace{10pt}

例如:
\begin{itemize}
    \item 今日は \textbf{昨日ほど} 暑くないです。
\end{itemize}

去掉「昨日ほど」后变为「今日は暑くないです。」,表示今天不热。\textbf{所以原句描述的是今天不热,加上比较的助词短语后,多了一层“比昨天”的含义。但仍然是今天不热。}
\vspace{10pt}

又如:
\begin{itemize}
    \item \textbf{日本より} 中国のほうが 広いです。
\end{itemize}

去掉「日本より」后,原句变为「中国が広いです」,说明大的是中国,而不是被助词修饰的「日本」。
\end{callout}