\chapter{標準日本語 第39課}

摘要:\textbf{無し}

\separatorline

\section{文法}

\subsection{「ところです」的各种意思}

\begin{codeblock}
・・・    ところです

・・・ている ところです

・・・た   ところです
\end{codeblock}

根据前接动词的形式,有不同的意思。

\textbf{1. 动词普通形 + ところです}

表示“即将・・・”“就要・・・”的意思,动作尚未开始,现在就要开始。

\textbf{2. 动词ている + ところです}

表示“正在・・・”的意思,表示某种动作正在进行之中。

\textbf{3. 动词た形 + ところです}

表示“刚刚・・・”的意思,动作刚刚结束。

\begin{example}
    \begin{itemize}
        \item ご飯を \textbf{食べる} ところです。   (\textbf{要}吃饭了)
        \item ご飯を \textbf{食べている} ところです。 (\textbf{正在}吃饭)
        \item ご飯を \textbf{食べた} ところです。   (\textbf{刚}吃完饭)
    \end{itemize}
\end{example}

\textbf{注:} 本语法看似与单独的「普通形/ている/た形」意思一样,但实则有不同。本语法强调的是\textbf{动作的三个阶段}:要做/正在做/刚做完。

\begin{callout}
    例如,老师问你paper进行的怎么样了,回答“正在收集资料”,应该说:「\textbf{資料を集めているところです}」,表示\textbf{正处在收集资料的阶段}。而不应该用「資料を集めているです」。因为你现在的动作不是在收集资料,而是在与老师对话。
\end{callout}

\separatorline

\subsection{「はず」表可能性、应该}

\begin{codeblock}
[・・・动词普通形] はずです

[・・・动词普通形] はずがない(否定形式)
\end{codeblock}

表示理所当然,是\textbf{说话人根据事物发展趋势做出的推测},相当于\textbf{“应该・・・”}。说话人对此有七八成的把握。

「はず」可以\textbf{看作一个名词},表可能性。则其\textbf{接续规则与名词一致}:[动词普通形 / 一类形容词 / 形容动词\textbf{な} / 名词\textbf{の}] + はず。

\begin{example}
    \begin{itemize}
        \item もうすぐ、 田中さんが \textbf{来る はずです}。
        \item 大学は 来週から 休みに\textbf{なる はずです}。
        \item ほとんどの人が、手帳を 持って\textbf{いる はずです}。
        \item 彼が そんなことを\textbf{する はずがありません}。(他\textbf{不可能}做那种事。)
    \end{itemize}
\end{example}

\separatorline

\section{词语与用法}

\subsection{记日记/笔记}

记日记:日記を\textbf{つける}

记笔记:メモを\textbf{取る}

\separatorline

\subsection{「について」表关于}

\begin{codeblock}
・・・について ~
\end{codeblock}

相当于"about・・・的~",作为一个修饰成分,表示\textbf{后项事物是关于前项的}。

如果后接名词,则用「・・・についての~」。如果后接动词句,则用「・・・について~」

\begin{example}
    \begin{itemize}
        \item 今 日本留学\textbf{についての} レポートを 書いているところです。
        
        (现在正在写\textbf{和}日本留学\textbf{有关}的报道)

    \end{itemize}
\end{example}

