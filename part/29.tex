\chapter{標準日本語 第29課}

摘要:\textbf{無し}

\separatorline

\section{文法}

\subsection{「くれる」表给(我)}

\begin{codeblock}
甲が 乙に ~を くれます
\end{codeblock}

表示\textbf{甲给乙}东西。此处,\textbf{乙一般是说话人}(或\textbf{说话人一方的人},关系和乙更近,例如乙的妹妹)。

如果乙是说话人,且是接受者,可以省略「乙に」。

\begin{callout}
    照任老师的话说,就是:\textbf{外は 内に}。

    \vspace{10pt}

    如果「内は 外に」,应该用「あげる」。
\end{callout}

\begin{example}
    \begin{itemize}
        \item 田中さんは \textbf{私に} 映画の切符を \textbf{くれました}。
        
        \item 誕生日に、 父は 時計を \textbf{くれました}。(给我。省略了「わたしに」,因为我是说话者。)
        
        \item 田中さんは \textbf{弟に} 本を \textbf{くれました}。
        
        \vspace{10pt}

        \item 私は \textbf{張さんに} 東京の地図を \textbf{あげました}。
    \end{itemize}
\end{example}

\separatorline

\newpage

\subsection{「こと」将动词句转换为名词句}

\begin{codeblock}
~は [・・・动词普通形]こと です
\end{codeblock}

使用「こと」可以将动词句转换为\textbf{具有相同意义的},\textbf{名词性质}的句子。

「こと」前应使用动词的\textbf{普通形}。

\begin{example}
    \begin{itemize}
        \item 私の 趣味は \textbf{テニスを すること} です。
        \item 彼女の 仕事は、 \textbf{小説を 書くこと} です。(她的工作是写小说。)
    \end{itemize}
\end{example}

\begin{callout}
    这个语法的核心在于:\textbf{动词句}通过「こと」变成了\textbf{名词句}。

    \vspace{10pt}

    所以,\textbf{动词句}可以作为\textbf{主语}、\textbf{宾语}等名词的位置出现。

    \vspace{10pt}

    「~は ・・・こと です」这个语法的本质是「~は ・・・です」。这只是一个例子。在其他语法结构里,也可以使用「こと」将动词句转换为名词句。
\end{callout}

\separatorline

\subsection{形容词的副词用法}

\begin{codeblock}
[・・・一类形词干]く

[・・・二类形词干]に
\end{codeblock}

形容词可以通过改变词尾,变成副词,用来修饰动词、形容词或其他副词。

\begin{example}
    \begin{itemize}
        \item 私は いつも \textbf{早く} 寝ます。
        \item 李さんは 東京で とても \textbf{楽しく} 過ごしました。
        \item 李さんは 田中さんに \textbf{丁寧に} お礼を 言いました。
        \item この箱は \textbf{静かに} 運んで ください。(请安静地搬运这个箱子)
    \end{itemize}
\end{example}

\begin{callout}
    这个语法的本质是把形容词变成了其\textbf{连用形}。连用形可以连接用言,从而修饰用言。关于连用形的内容,请参考附录。

    \vspace{10pt}

    一类形容词的连用形是把「い」变成「く」,二类形容词的连用形是把「だ」变成「に」。

    \vspace{10pt}

    名词 +「だ」的连用形也是把「だ」变成「に」。因此,名词也可以通过这种方式变成副词,用来修饰动词等。
    
\end{callout}