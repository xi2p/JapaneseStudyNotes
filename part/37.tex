\chapter{標準日本語 第37課}

摘要:\textbf{無し}

\separatorline

\section{可能态活用}



\begin{tabularx}{\linewidth}{|X|X|X|X|}
    \hline
    动词类型 & 变形规则 & 基本形举例 & 变形举例 \\
    \hline
    一类动词 & 结尾う段变え段+る & 話す & 話せる \\
    \hline
    二类动词 & 去结尾る+られる & 食べる & 食べられる \\
    \hline
    来る & 来(こ)られる & \ & \\
    \hline
    する & できる & \ & \\
    \hline
\end{tabularx}

将动词变形成可能态,\textbf{表示具备做某事的能力}。意思与「・・・ことができます」基本一致。

注意,原来“名词+を+动词”的结构要改成“名词+\textbf{が}+可能动词”。因为可能动词描述的是一种\textbf{状态}。

\begin{example}
    \begin{itemize}
        \item 田中さんは テニス\textbf{が できます}。
        \item 国際電話は 簡単に \textbf{かけられます}。(很简单就\textbf{能}打国际电话)
        \item 明日も ここに \textbf{来られます}か。(明天也\textbf{能}来吗)
    \end{itemize}
\end{example}

\begin{callout}
    「見られる」和「見える」都是解释为“能看见”,但有不同。
    
    \vspace{10pt}

    「見える」表示客观的能看见,自然地能映入眼帘,与说话人的意志无关。

    \vspace{10pt}

    「見られる」则表示\textbf{人有能力}看见,是意志、努力的结果。

    \vspace{10pt}

    「聞ける」「聞こえる」也是相同关系。
\end{callout}

\separatorline

\newpage

\section{文法}

\subsection{「ために」表目的}

\begin{codeblock}
[・・・动词基本形] ために、~

[・・・名词] のために、~
\end{codeblock}

表示\textbf{“为了・・・”}。

\begin{example}
    \begin{itemize}
        \item 3時のバスに \textbf{乗るために}、 二時半に 家を 出ました。

        \item 昔の人は 情報を \textbf{伝えるために}、 鳥を 使います。

        \item \textbf{技術開発のために}、 多くの 科学者たちが 研究を 進めています。
    \end{itemize}
\end{example}

\separatorline

\subsection{「ようにします」表为实现目标而努力}

\begin{codeblock}
・・・ようにします
\end{codeblock}

表示\textbf{为了实现“・・・”所述目标而努力},为此用心,为此而努力的意思。可以理解成\textbf{“争取・・・” “力争・・・”}。

“・・・”处动词用基本形或ない形。

\begin{example}
    \begin{itemize}
        \item 明日から もっと 早く \textbf{起きるようにします}。(明天开始\textbf{努力}早点起)
        \item 授業に \textbf{遅れないようにします}。(\textbf{争取}上课不要迟到)
        \item 電話を かけるのを、 \textbf{忘れないようにします}。
    \end{itemize}
\end{example}

\separatorline

\subsection{「ようになります」表变得}

\begin{codeblock}
・・・ようになります
\end{codeblock}

与语法3不同,此处表示事物自然发展变化。\textbf{此语法强调一种“变化感”。“变得”}。

\begin{callout}
    「よう」的汉字写作“様”,表示“样子、状态”的意思。

    \vspace{10pt}

    所以「ようになります」可以理解为“变成・・・那样”,即“变得・・・”。
\end{callout}

“・・・”处动词用基本形或ない形。

\begin{example}
    \begin{itemize}
        \item この本を 勉強すれば、 日本語が \textbf{話せるようになります}。
        
        (如果学了这本书,就会\textbf{变得}会说日语)

        \item テニスが \textbf{できるようになりました}。
        
        (\textbf{学会}网球\textbf{了})
    \end{itemize}
\end{example}

\separatorline

\section{词语与用法}

\subsection{「実は」と「実に」}

「実は」意思是“其实・・・”,常用于说出请求对方做某事的场合。

「実に」表示程度高,程度略低于「とても」
