\chapter{標準日本語 第20課}

摘要:\textbf{动词基本形}

\separatorline

\section{文法}

\subsection{「できる」表会做某事}

\begin{codeblock}
・・・が できます
\end{codeblock}

表示有能力做某事。\textbf{「が」前接名词}。



\begin{example}
    \begin{itemize}
        \item わたしは \textbf{英語が できます}。
    \end{itemize}
\end{example}

\separatorline

\subsection{「ことができる」表会做某事}

\begin{codeblock}
[・・・动词基本形] ことが できます
\end{codeblock}

表示有能力做某事。\textbf{「こと」前接动词基本形}。



\begin{example}
    \begin{itemize}
        \item 餃子を \textbf{作ることが できます}。
        \item 田中さんは 英語を \textbf{話すことが できます}。
    \end{itemize}

\end{example}

\separatorline

\subsection{「前に」表在做・・・之前,~}

\begin{codeblock}
[・・・动词基本形] 前に、~
\end{codeblock}

注意,「前に」前接动词基本形。也就是说,\textbf{「前に」前的动词与时态无关},不会出现过去式。



\begin{example}
    \begin{itemize}
        \item 会社に \textbf{行く 前に} 電話をします。
        \item 料理を \textbf{作る 前に} 手を洗います。
    \end{itemize}
\end{example}

