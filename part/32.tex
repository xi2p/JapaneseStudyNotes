\chapter{標準日本語 第32課}

摘要:\textbf{自他动词}

\separatorline

\section{文法}

\subsection{「てある」表处于某种状态}

\begin{codeblock}
[・・・て] あります
\end{codeblock}

\textbf{他动词}加上「てあります」,表示事物\textbf{处于某种状态},并且\textbf{强调了这个动作是某人做的,且做完了的}。

\begin{codeblock}
    虽然是他动词,但此语法下,\textbf{应使用「が」提示他动词作用的对象}。因为此语法表示的是\textbf{状态},他动词作用的对象成了主语。
\end{codeblock}

\begin{example}
    \begin{itemize}
        \item 冷蔵庫に 肉やビールが \textbf{入れて あります}。 (把肉和啤酒放到冰箱里了)
        \item 今日は 寒いので、 ストーブが \textbf{つけて あります}。
        \item テーブルの上に かばんが \textbf{置いて あります}。
    \end{itemize}
\end{example}

\separatorline

\subsection{「ている」表处于某种状态}

\begin{codeblock}
[・・・て] います
\end{codeblock}

\textbf{自动词}加上「ています」,表示事物处于某种状态。

\begin{example}
    \begin{itemize}
        \item 部屋の窓が \textbf{開いて います}。(房间的窗户开着)
        \item 電話が \textbf{鳴って います}。
        \item 冷蔵庫に 肉やビールが \textbf{入って います}。
    \end{itemize}
\end{example}

\begin{callout}
「てある」与「ている」相比,多了一层\textbf{强调某人做了这个动作}的意思:
\begin{itemize}
    \item 冷蔵庫に 肉やビールが 入れて あります。(某人)(\textbf{把}肉和啤酒放到冰箱里\textbf{了})
    \item 冷蔵庫に 肉やビールが 入って います。 (肉在冰箱里了)
\end{itemize}

\end{callout}

\separatorline

\subsection{「てみる」表尝试做某事}

\begin{codeblock}
[・・・て] みます
\end{codeblock}

表示表示试试看做某事。

\begin{example}
    \begin{itemize}
        \item 日本の映画を \textbf{見て みました}。 とても面白かったです。
        \item 張さんは 初めて 日本語の年賀状を \textbf{書いて みました}。
    \end{itemize}
\end{example}

\separatorline

\section{词语与用法}
\subsection{「なかなか」表不容易}

副词,表示\textbf{不容易}。带有一层\textbf{“尽管付出了努力或期待,但依然难以实现”}的意思。

使用时,\textbf{其后伴随动词的否定形式},表示某种状态不易产生。

\begin{example}
    \begin{itemize}
        \item 窓が \textbf{なかなか} 開き\textbf{ません}。(窗户怎么也打不开)
        
        (使用自动词「開く」强调因窗户自身属性而打不开)

        \item 初めてので、 \textbf{なかなか} 上手に でき\textbf{ません}。(因为是第一次,所以怎么也做不好)
    \end{itemize}
\end{example}

\separatorline

\subsection{自动词与他动词背诵辨别的简单方法}

一般来说,日语自动词和他动词是成对出现的。他们表达相似的意思,且单词写法相似,例如:

\begin{itemize}
    \item ドアが 開く (自动词)/ ドアを 開ける (他动词)
    \item 電気が つく (自动词)/ 電気を つける (他动词)
    \item 机が 壊れる (自动词)/ 机を  壊す  (他动词)
\end{itemize}

不难发现一些规律(\textbf{以下规律并非绝对,仍有例外情况。}):

当你只看到一个单词时,若其结尾是「す」或结尾罗马音"eru",则有可能是\textbf{他动词};

当你看到成对的自他单词时,其中某一个单词是他动词的概率:\textbf{结尾是「す」> 结尾罗马音"eru" > 其他结尾}。

\begin{callout}
    例如,你只看到单词「閉める」,因为其结尾是"eru",所以有可能它是他动词。

    当你看到「壊れる」、「壊す」这对自他单词时,因为「壊す」结尾是「す」,概率更大,所以它是他动词,那么「壊れる」就是自动词(尽管他概率也很大)。

    \textbf{若你记住了一组自他动词,但不记得谁是自动、谁是他动,可以用这个方法来猜测。}
\end{callout}
