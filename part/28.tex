\chapter{標準日本語 第28課}

摘要:\textbf{推量形}

\separatorline

\section{推量形活用}

\begin{tabularx}{\linewidth}{|X|X|X|X|}
\hline
动词类型 & 变形规律 & 原型示例 & 变形示例 \\
\hline
一类动词 & 结尾变お行 + う & 読む & 読もう \\
\hline
二类动词 & 结尾去る + よう & 食べる & 食べよう \\
\hline
する & しよう & \ & \ \\
\hline
くる & 来(こ)よう & \ & \ \\
\hline
\end{tabularx}

\separatorline

\section{文法}

\subsection{推量形打算做}

\begin{codeblock}
[・・・推量形]と 思います/思っています
\end{codeblock}

表示有计划地打算做某件事。

\begin{example}
    \begin{itemize}
        \item 夏休みは 海へ \textbf{行こうと 思います}。
        \item 李さんは 秋葉原で ラジオカセットを \textbf{買おうと 思っています}。
        \item 李さんは ラジオカセットを 息子さんへの お土産に \textbf{しようと 思っています}。
    \end{itemize}
\end{example}

\separatorline

\newpage

\subsection{「ので」表原因}

\begin{codeblock}
・・・ので、・・・
\end{codeblock}

表示前句是后句的原因或理由。

依照「ので」前面的词的不同,有不同的接续要求:

\begin{tabularx}{\linewidth}{|X|X|X|X|}
\hline
词性 & 接续方式 & 示例词 & 示例句 \\
\hline
动词 & \textbf{普通形}+ので & 降りました(降る) & 降ったので \\
\hline
一类形容词 & \textbf{普通形}+ので & 痛い & 痛いので \\
\hline
形容动词 & \textbf{词干+な}+ので & 簡単 & 簡単なので \\
\hline
名词 & \textbf{名词+な}+ので & 休み & 休みなので \\
\hline
\end{tabularx}

\begin{example}
    \begin{itemize}
        \item 頭が \textbf{痛いので}、 仕事を 休みました。

        \item 日曜日は、 雨が \textbf{降ったので}、 出かけませんでした。

        \item \textbf{病気なので}、 仕事を 休んでいます。

    \end{itemize}
\end{example}

\separatorline

\subsection{「かもしれません」表推测}

\begin{codeblock}
・・・かもしれません
・・・かもしれない(口语形式)
\end{codeblock}

表推测,也许,可能。说话人不能确定的程度相当大。

依照「かもしれません」前面的词的不同,有不同的接续要求:

\begin{tabularx}{\linewidth}{|X|X|}
\hline
词性 & 接续方式 \\
\hline
动词 & \textbf{普通形}+かもしれません \\
\hline
一类形容词 & \textbf{普通形}+かもしれません \\
\hline
形容动词 & \textbf{词干}+かもしれません \\
\hline
名词 & \textbf{名词}+かもしれません \\
\hline
\end{tabularx}

\begin{example}
    \begin{itemize}
        \item 空が 暗いですから、 午後から 雨が \textbf{降る かもしれません}。
        \item 渋滞しているので、 30分以上 \textbf{かかる かもしれません}。
        \item 山の上は \textbf{寒い かもしれません}。
        \item 日曜日なので、 \textbf{休み かもしれません}。
    \end{itemize}
\end{example}
