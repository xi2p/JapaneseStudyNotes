\chapter{標準日本語 第14課}
 
摘要:\textbf{想要,表目的}

\separatorline

\section{文法}

\subsection{「ほしい」表示想要某物}
\begin{codeblock}
甲(人称)は + 乙(物品)が + ほしいです
\end{codeblock}
表示\textbf{想要某物}。此句型只用于描述自己的欲望,或询问他人的欲望。不能用于描述第三者的欲望。
「ほしい」本身可以看做一类形容词,将其变成否定形、过去形等,可以表达过去、否定等含义。

\begin{example}
    \begin{itemize}
        \item わたしは \textbf{本が ほしい}です
    \end{itemize}
\end{example}

\separatorline

\subsection{「たい」表示想做某事}
\begin{codeblock}
甲(人称)は + 乙(对象)が + 动词连用形 + たいです
\end{codeblock}
表示\textbf{想做某事}。此句型只用于描述自己的欲望,或询问他人的欲望。不能用于描述第三者的欲望。
「たい」本身可以看做一类形容词,将其变成否定形、过去形等,可以表达过去、否定等含义。

\begin{example}
    \begin{itemize}
        \item わたしは 本が \textbf{読み たい} です
        \item わたしは \textbf{遊び たくない} です
        \item わたしは ここへ \textbf{行き たかった} です
        \item わたしは \textbf{寝 たくなかった} です。(我之前不想睡觉。)
    \end{itemize}
\end{example}

\separatorline

\subsection{去某地做某事}
\begin{codeblock}
场所へ + (乙を) + 动词连用形 + に + 行きます/来ます
\end{codeblock}
表示去某地做某事。\textbf{「に」表示目的}。为了某个目的“前往/来某地”。

\begin{example}
    \begin{itemize}
        \item 図書館へ 本を \textbf{借りに 行きます。}(去图书馆借书。)
    \end{itemize}
\end{example}